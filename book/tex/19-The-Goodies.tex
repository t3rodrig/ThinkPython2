\chapter{The Goodies}

One of my goals for this book has been to teach you as little Python
as possible.  When there were two ways to do something, I picked 
one and avoided mentioning the other.  Or sometimes I put the second
one into an exercise.

Now I want to go back for some of the good bits that got left behind.
Python provides a number of features that are not really necessary---you
can write good code without them---but with them you can sometimes
write code that's more concise, readable or efficient, and sometimes
all three.

% TODO: add the with statement

\section{Conditional expressions}

We saw conditional statements in Section~\ref{conditional.execution}.
Conditional statements are often used to choose one of two values;
for example:
\index{conditional expression}
\index{expression!conditional}

\begin{verbatim}
if x > 0:
    y = math.log(x)
else:
    y = float('nan')
\end{verbatim}

This statement checks whether {\tt x} is positive.  If so, it computes
{\tt math.log}.  If not, {\tt math.log} would raise a ValueError.  To
avoid stopping the program, we generate a ``NaN'', which is a special
floating-point value that represents ``Not a Number''.
\index{NaN}
\index{floating-point}

We can write this statement more concisely using a {\bf conditional
expression}:

\begin{verbatim}
y = math.log(x) if x > 0 else float('nan')
\end{verbatim}

You can almost read this line like English: ``{\tt y} gets log-{\tt x}
if {\tt x} is greater than 0; otherwise it gets NaN''.

Recursive functions can sometimes be rewritten using conditional
expressions.  For example, here is a recursive version of {\tt factorial}:
\index{factorial}
\index{function!factorial}

\begin{verbatim}
def factorial(n):
    if n == 0:
        return 1
    else:
        return n * factorial(n-1)
\end{verbatim}

We can rewrite it like this:

\begin{verbatim}
def factorial(n):
    return 1 if n == 0 else n * factorial(n-1)
\end{verbatim}

Another use of conditional expressions is handling optional
arguments.  For example, here is the init method from
{\tt GoodKangaroo} (see Exercise~\ref{kangaroo}):
\index{optional argument}
\index{argument!optional}

\begin{verbatim}
    def __init__(self, name, contents=None):
        self.name = name
        if contents == None:
            contents = []
        self.pouch_contents = contents
\end{verbatim}

We can rewrite this one like this:

\begin{verbatim}
    def __init__(self, name, contents=None):
        self.name = name
        self.pouch_contents = [] if contents == None else contents 
\end{verbatim}

In general, you can replace a conditional statement with a conditional
expression if both branches contain simple expressions that are
either returned or assigned to the same variable.
\index{conditional statement}
\index{statement!conditional}



\section{List comprehensions}

In Section~\ref{filter} we saw the map and filter patterns.  For
example, this function takes a list of strings, maps the string method
{\tt capitalize} to the elements, and returns a new list of strings:

\begin{verbatim}
def capitalize_all(t):
    res = []
    for s in t:
        res.append(s.capitalize())
    return res
\end{verbatim}

We can write this more concisely using a {\bf list comprehension}:
\index{list comprehension}

\begin{verbatim}
def capitalize_all(t):
    return [s.capitalize() for s in t]
\end{verbatim}

The bracket operators indicate that we are constructing a new
list.  The expression inside the brackets specifies the elements
of the list, and the {\tt for} clause indicates what sequence
we are traversing.
\index{list}
\index{for loop}

The syntax of a list comprehension is a little awkward because
the loop variable, {\tt s} in this example, appears in the expression
before we get to the definition.
\index{loop variable}

List comprehensions can also be used for filtering.  For example,
this function selects only the elements of {\tt t} that are
upper case, and returns a new list:
\index{filter pattern}
\index{pattern!filter}

\begin{verbatim}
def only_upper(t):
    res = []
    for s in t:
        if s.isupper():
            res.append(s)
    return res
\end{verbatim}

We can rewrite it using a list comprehension

\begin{verbatim}
def only_upper(t):
    return [s for s in t if s.isupper()]
\end{verbatim}

List comprehensions are concise and easy to read, at least for simple
expressions.  And they are usually faster than the equivalent for
loops, sometimes much faster.  So if you are mad at me for not
mentioning them earlier, I understand.

But, in my defense, list comprehensions are harder to debug because
you can't put a print statement inside the loop.  I suggest that you
use them only if the computation is simple enough that you are likely
to get it right the first time.  And for beginners that means never.
\index{debugging}



\section{Generator expressions}

{\bf Generator expressions} are similar to list comprehensions, but
with parentheses instead of square brackets:
\index{generator expression}
\index{expression!generator}

\begin{verbatim}
>>> g = (x**2 for x in range(5))
>>> g
<generator object <genexpr> at 0x7f4c45a786c0>
\end{verbatim}
%
The result is a generator object that knows how to iterate through
a sequence of values.  But unlike a list comprehension, it does not
compute the values all at once; it waits to be asked.  The built-in
function {\tt next} gets the next value from the generator:
\index{generator object}
\index{object!generator}

\begin{verbatim}
>>> next(g)
0
>>> next(g)
1
\end{verbatim}
%
When you get to the end of the sequence, {\tt next} raises a 
StopIteration exception.  You can also use a {\tt for} loop to iterate
through the values:
\index{StopIteration}
\index{exception!StopIteration}

\begin{verbatim}
>>> for val in g:
...     print(val)
4
9
16
\end{verbatim}
%
The generator object keeps track of where it is in the sequence,
so the {\tt for} loop picks up where {\tt next} left off.  Once the
generator is exhausted, it continues to raise {\tt StopIteration}:

\begin{verbatim}
>>> next(g)
StopIteration
\end{verbatim}

Generator expressions are often used with functions like {\tt sum},
{\tt max}, and {\tt min}:
\index{sum}
\index{function!sum}

\begin{verbatim}
>>> sum(x**2 for x in range(5))
30
\end{verbatim}


\section{{\tt any} and {\tt all}}

Python provides a built-in function, {\tt any}, that takes a sequence
of boolean values and returns {\tt True} if any of the values are {\tt
  True}.  It works on lists:
\index{any}
\index{built-in function!any}

\begin{verbatim}
>>> any([False, False, True])
True
\end{verbatim}
%
But it is often used with generator expressions:
\index{generator expression}
\index{expression!generator}

\begin{verbatim}
>>> any(letter == 't' for letter in 'monty')
True
\end{verbatim}
%
That example isn't very useful because it does the same thing
as the {\tt in} operator.  But we could use {\tt any} to rewrite
some of the search functions we wrote in Section~\ref{search}.  For
example, we could write {\tt avoids} like this:
\index{search pattern}
\index{pattern!search}

\begin{verbatim}
def avoids(word, forbidden):
    return not any(letter in forbidden for letter in word)
\end{verbatim}
%
The function almost reads like English, ``{\tt word} avoids
{\tt forbidden} if there are not any forbidden letters in {\tt word}.''

Using {\tt any} with a generator expression is efficient because
it stops immediately if it finds a {\tt True} value,
so it doesn't have to evaluate the whole sequence.

Python provides another built-in function, {\tt all}, that returns
{\tt True} if every element of the sequence is {\tt True}.  As
an exercise, use {\tt all} to re-write \verb"uses_all" from
Section~\ref{search}.
\index{all}
\index{built-in function!any}


\section{Sets}
\label{sets}

In Section~\ref{dictsub} I use dictionaries to find the words
that appear in a document but not in a word list.  The function
I wrote takes {\tt d1}, which contains the words from the document
as keys, and {\tt d2}, which contains the list of words.  It
returns a dictionary that contains the keys from {\tt d1} that
are not in {\tt d2}.

\begin{verbatim}
def subtract(d1, d2):
    res = dict()
    for key in d1:
        if key not in d2:
            res[key] = None
    return res
\end{verbatim}
%
In all of these dictionaries, the values are {\tt None} because
we never use them.  As a result, we waste some storage space.
\index{dictionary subtraction}

Python provides another built-in type, called a {\tt set}, that
behaves like a collection of dictionary keys with no values.  Adding
elements to a set is fast; so is checking membership.  And sets
provide methods and operators to compute common set operations.
\index{set}
\index{object!set}

For example, set subtraction is available as a method called
{\tt difference} or as an operator, {\tt -}.  So we can rewrite
{\tt subtract} like this:
\index{set subtraction}

\begin{verbatim}
def subtract(d1, d2):
    return set(d1) - set(d2)
\end{verbatim}
%
The result is a set instead of a dictionary, but for operations like
iteration, the behavior is the same.

Some of the exercises in this book can be done concisely and
efficiently with sets.  For example, here is a solution to
\verb"has_duplicates", from
Exercise~\ref{duplicate}, that uses a dictionary:

\begin{verbatim}
def has_duplicates(t):
    d = {}
    for x in t:
        if x in d:
            return True
        d[x] = True
    return False
\end{verbatim}

When an element appears for the first time, it is added to the
dictionary.  If the same element appears again, the function returns
{\tt True}.

Using sets, we can write the same function like this:

\begin{verbatim}
def has_duplicates(t):
    return len(set(t)) < len(t)
\end{verbatim}
%
An element can only appear in a set once, so if an element in {\tt t}
appears more than once, the set will be smaller than {\tt t}.  If there
are no duplicates, the set will be the same size as {\tt t}.
\index{duplicate}

We can also use sets to do some of the exercises in
Chapter~\ref{wordplay}.  For example, here's a version of
\verb"uses_only" with a loop:

\begin{verbatim}
def uses_only(word, available):
    for letter in word: 
        if letter not in available:
            return False
    return True
\end{verbatim}
%
\verb"uses_only" checks whether all letters in {\tt word} are
in {\tt available}.  We can rewrite it like this:

\begin{verbatim}
def uses_only(word, available):
    return set(word) <= set(available)
\end{verbatim}
%
The \verb"<=" operator checks whether one set is a subset or another,
including the possibility that they are equal, which is true if all
the letters in {\tt word} appear in {\tt available}.
\index{subset}

As an exercise, rewrite \verb"avoids" using sets.


\section{Counters}

A Counter is like a set, except that if an element appears more
than once, the Counter keeps track of how many times it appears.
If you are familiar with the mathematical idea of a {\bf multiset},
a Counter is a natural way to represent a multiset.
\index{Counter}
\index{object!Counter}
\index{multiset}

Counter is defined in a standard module called {\tt collections},
so you have to import it.  You can initialize a Counter with a string,
list, or anything else that supports iteration:
\index{collections}
\index{module!collections}

\begin{verbatim}
>>> from collections import Counter
>>> count = Counter('parrot')
>>> count
Counter({'r': 2, 't': 1, 'o': 1, 'p': 1, 'a': 1})
\end{verbatim}

Counters behave like dictionaries in many ways; they map from each
key to the number of times it appears.  As in dictionaries,
the keys have to be hashable.

Unlike dictionaries, Counters don't raise an exception if you access
an element that doesn't appear.  Instead, they return 0:

\begin{verbatim}
>>> count['d']
0
\end{verbatim}

We can use Counters to rewrite \verb"is_anagram" from
Exercise~\ref{anagram}:

\begin{verbatim}
def is_anagram(word1, word2):
    return Counter(word1) == Counter(word2)
\end{verbatim}

If two words are anagrams, they contain the same letters with the same
counts, so their Counters are equivalent.

Counters provide methods and operators to perform set-like operations,
including addition, subtraction, union and intersection.  And
they provide an often-useful method, \verb"most_common", which
returns a list of value-frequency pairs, sorted from most common to
least:

\begin{verbatim}
>>> count = Counter('parrot')
>>> for val, freq in count.most_common(3):
...     print(val, freq)
r 2
p 1
a 1
\end{verbatim}


\section{defaultdict}

The {\tt collections} module also provides {\tt defaultdict}, which is
like a dictionary except that if you access a key that doesn't exist,
it can generate a new value on the fly.
\index{defaultdict}
\index{object!defaultdict}
\index{collections}
\index{module!collections}

When you create a defaultdict, you provide a function that's used to
create new values.  A function used to create objects is sometimes
called a {\bf factory}.  The built-in functions that create lists, sets,
and other types can be used as factories:
\index{factory function}

\begin{verbatim}
>>> from collections import defaultdict
>>> d = defaultdict(list)
\end{verbatim}

Notice that the argument is {\tt list}, which is a class object,
not {\tt list()}, which is a new list.  The function you provide
doesn't get called unless you access a key that doesn't exist.

\begin{verbatim}
>>> t = d['new key']
>>> t
[]
\end{verbatim}

The new list, which we're calling {\tt t}, is also added to the
dictionary.  So if we modify {\tt t}, the change appears in {\tt d}:

\begin{verbatim}
>>> t.append('new value')
>>> d
defaultdict(<class 'list'>, {'new key': ['new value']})
\end{verbatim}

If you are making a dictionary of lists, you can often write simpler
code using {\tt defaultdict}.  In my solution to
Exercise~\ref{anagrams}, which you can get from
\url{http://thinkpython2.com/code/anagram_sets.py}, I make a
dictionary that maps from a sorted string of letters to the list of
words that can be spelled with those letters.  For example, {\tt
  'opst'} maps to the list {\tt ['opts', 'post', 'pots', 'spot',
    'stop', 'tops']}.

Here's the original code:

\begin{verbatim}
def all_anagrams(filename):
    d = {}
    for line in open(filename):
        word = line.strip().lower()
        t = signature(word)
        if t not in d:
            d[t] = [word]
        else:
            d[t].append(word)
    return d
\end{verbatim}

This can be simplified using {\tt setdefault}, which you might
have used in Exercise~\ref{setdefault}:
\index{setdefault}

\begin{verbatim}
def all_anagrams(filename):
    d = {}
    for line in open(filename):
        word = line.strip().lower()
        t = signature(word)
        d.setdefault(t, []).append(word)
    return d
\end{verbatim}

This solution has the drawback that it makes a new list
every time, regardless of whether it is needed.  For lists,
that's no big deal, but if the factory
function is complicated, it might be.
\index{factory function}

We can avoid this problem and 
simplify the code using a {\tt defaultdict}:

\begin{verbatim}
def all_anagrams(filename):
    d = defaultdict(list)
    for line in open(filename):
        word = line.strip().lower()
        t = signature(word)
        d[t].append(word)
    return d
\end{verbatim}

My solution to Exercise~\ref{poker}, which you can download from
\url{http://thinkpython2.com/code/PokerHandSoln.py},
uses {\tt setdefault} in the function
\verb"has_straightflush".  This solution has the drawback
of creating a {\tt Hand} object every time through the loop, whether
it is needed or not.  As an exercise, rewrite it using
a defaultdict.


\section{Named tuples}

Many simple objects are basically collections of related values.
For example, the Point object defined in Chapter~\ref{clobjects} contains
two numbers, {\tt x} and {\tt y}.  When you define a class like
this, you usually start with an init method and a str method:

\begin{verbatim}
class Point:

    def __init__(self, x=0, y=0):
        self.x = x
        self.y = y

    def __str__(self):
        return '(%g, %g)' % (self.x, self.y)
\end{verbatim}

This is a lot of code to convey a small amount of information.
Python provides a more concise way to say the same thing:

\begin{verbatim}
from collections import namedtuple
Point = namedtuple('Point', ['x', 'y'])
\end{verbatim}

The first argument is the name of the class you want to create.
The second is a list of the attributes Point objects should have,
as strings.  The return value from {\tt namedtuple} is a class object:
\index{namedtuple}
\index{object!namedtuple}
\index{collections}
\index{module!collections}

\begin{verbatim}
>>> Point
<class '__main__.Point'>
\end{verbatim}

{\tt Point} automatically provides methods like \verb"__init__" and
\verb"__str__" so you don't have to write them.
\index{class object}
\index{object!class}

To create a Point object, you use the Point class as a function:

\begin{verbatim}
>>> p = Point(1, 2)
>>> p
Point(x=1, y=2)
\end{verbatim}

The init method assigns the arguments to attributes using the names
you provided.  The str method prints a representation of the Point
object and its attributes.

You can access the elements of the named tuple by name:

\begin{verbatim}
>>> p.x, p.y
(1, 2)
\end{verbatim}

But you can also treat a named tuple as a tuple:

\begin{verbatim}
>>> p[0], p[1]
(1, 2)

>>> x, y = p
>>> x, y
(1, 2)
\end{verbatim}

Named tuples provide a quick way to define simple classes.
The drawback is that simple classes don't always stay simple.
You might decide later that you want to add methods to a named tuple.
In that case, you could define a new class that inherits from
the named tuple:
\index{inheritance}

\begin{verbatim}
class Pointier(Point):
    # add more methods here
\end{verbatim}

Or you could switch to a conventional class definition.


\section{Gathering keyword args}

In Section~\ref{gather}, we saw how to write a function that
gathers its arguments into a tuple:
\index{gather}

\begin{verbatim}
def printall(*args):
    print(args)
\end{verbatim}
%
You can call this function with any number of positional arguments
(that is, arguments that don't have keywords):
\index{positional argument}
\index{argument!positional}

\begin{verbatim}
>>> printall(1, 2.0, '3')
(1, 2.0, '3')
\end{verbatim}
%
But the {\tt *} operator doesn't gather keyword arguments:
\index{keyword argument}
\index{argument!keyword}

\begin{verbatim}
>>> printall(1, 2.0, third='3')
TypeError: printall() got an unexpected keyword argument 'third'
\end{verbatim}
%
To gather keyword arguments, you can use the {\tt **} operator:

\begin{verbatim}
def printall(*args, **kwargs):
    print(args, kwargs)
\end{verbatim}
%
You can call the keyword gathering parameter anything you want, but
{\tt kwargs} is a common choice.  The result is a dictionary that maps
keywords to values:

\begin{verbatim}
>>> printall(1, 2.0, third='3')
(1, 2.0) {'third': '3'}
\end{verbatim}
%
If you have a dictionary of keywords and values, you can use the
scatter operator, {\tt **} to call a function:
\index{scatter}

\begin{verbatim}
>>> d = dict(x=1, y=2)
>>> Point(**d)
Point(x=1, y=2)
\end{verbatim}
%
Without the scatter operator, the function would treat {\tt d} as
a single positional argument, so it would assign {\tt d} to
{\tt x} and complain because there's nothing to assign to {\tt y}:

\begin{verbatim}
>>> d = dict(x=1, y=2)
>>> Point(d)
Traceback (most recent call last):
  File "<stdin>", line 1, in <module>
TypeError: __new__() missing 1 required positional argument: 'y'
\end{verbatim}
%
When you are working with functions that have a large number of
parameters, it is often useful to create and pass around dictionaries
that specify frequently used options.


\section{Glossary}

\begin{description}

\item[conditional expression:] An expression that has one of two
values, depending on a condition.
\index{conditional expression}
\index{expression!conditional}

\item[list comprehension:] An expression with a {\tt for} loop in square
brackets that yields a new list.
\index{list comprehension}

\item[generator expression:] An expression with a {\tt for} loop in parentheses
that yields a generator object.  
\index{generator expression}
\index{expression!generator}

\item[multiset:] A mathematical entity that represents a mapping
between the elements of a set and the number of times they appear.

\item[factory:] A function, usually passed as a parameter, used to
create objects. 
\index{factory}

\end{description}




\section{Exercises}

\begin{exercise}

The following is a function computes the binomial
coefficient recursively.

\begin{verbatim}
def binomial_coeff(n, k):
    """Compute the binomial coefficient "n choose k".

    n: number of trials
    k: number of successes

    returns: int
    """
    if k == 0:
        return 1
    if n == 0:
        return 0

    res = binomial_coeff(n-1, k) + binomial_coeff(n-1, k-1)
    return res
\end{verbatim}

Rewrite the body of the function using nested conditional
expressions.

One note: this function is not very efficient because it ends up computing
the same values over and over.  You could make it more efficient by
memoizing (see Section~\ref{memoize}).  But you will find that it's harder to
memoize if you write it using conditional expressions.

\end{exercise}
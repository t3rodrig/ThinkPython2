\chapter{Tuples}
\label{tuplechap}

This chapter presents one more built-in type, the tuple, and then
shows how lists, dictionaries, and tuples work together.
I also present a useful feature for variable-length argument lists,
the gather and scatter operators.

One note: there is no consensus on how to pronounce ``tuple''.
Some people say ``tuh-ple'', which rhymes with ``supple''.  But
in the context of programming, most people say ``too-ple'', which
rhymes with ``quadruple''.


\section{Tuples are immutable}
\index{tuple}
\index{type!tuple}
\index{sequence}

A tuple is a sequence of values.  The values can be any type, and
they are indexed by integers, so in that respect tuples are a lot
like lists.  The important difference is that tuples are immutable.
\index{mutability}
\index{immutability}

Syntactically, a tuple is a comma-separated list of values:

\begin{verbatim}
>>> t = 'a', 'b', 'c', 'd', 'e'
\end{verbatim}
%
Although it is not necessary, it is common to enclose tuples in
parentheses:
\index{parentheses!tuples in}

\begin{verbatim}
>>> t = ('a', 'b', 'c', 'd', 'e')
\end{verbatim}
%
To create a tuple with a single element, you have to include a final
comma:
\index{singleton}
\index{tuple!singleton}

\begin{verbatim}
>>> t1 = 'a',
>>> type(t1)
<class 'tuple'>
\end{verbatim}
%
A value in parentheses is not a tuple:

\begin{verbatim}
>>> t2 = ('a')
>>> type(t2)
<class 'str'>
\end{verbatim}
%
Another way to create a tuple is the built-in function {\tt tuple}.
With no argument, it creates an empty tuple:
\index{tuple function}
\index{function!tuple}

\begin{verbatim}
>>> t = tuple()
>>> t
()
\end{verbatim}
%
If the argument is a sequence (string, list or tuple), the result
is a tuple with the elements of the sequence:

\begin{verbatim}
>>> t = tuple('lupins')
>>> t
('l', 'u', 'p', 'i', 'n', 's')
\end{verbatim}
%
Because {\tt tuple} is the name of a built-in function, you should
avoid using it as a variable name.

Most list operators also work on tuples.  The bracket operator
indexes an element:
\index{bracket operator}
\index{operator!bracket}

\begin{verbatim}
>>> t = ('a', 'b', 'c', 'd', 'e')
>>> t[0]
'a'
\end{verbatim}
%
And the slice operator selects a range of elements.
\index{slice operator}
\index{operator!slice}
\index{tuple!slice}
\index{slice!tuple}

\begin{verbatim}
>>> t[1:3]
('b', 'c')
\end{verbatim}
%
But if you try to modify one of the elements of the tuple, you get
an error:
\index{exception!TypeError}
\index{TypeError}
\index{item assignment}
\index{assignment!item}

\begin{verbatim}
>>> t[0] = 'A'
TypeError: object doesn't support item assignment
\end{verbatim}
%
Because tuples are immutable, you can't modify the elements.  But you
can replace one tuple with another:

\begin{verbatim}
>>> t = ('A',) + t[1:]
>>> t
('A', 'b', 'c', 'd', 'e')
\end{verbatim}
%
This statement makes a new tuple and then makes {\tt t} refer to it.

The relational operators work with tuples and other sequences;
Python starts by comparing the first element from each
sequence.  If they are equal, it goes on to the next elements,
and so on, until it finds elements that differ.  Subsequent
elements are not considered (even if they are really big).
\index{comparison!tuple}
\index{tuple!comparison}

\begin{verbatim}
>>> (0, 1, 2) < (0, 3, 4)
True
>>> (0, 1, 2000000) < (0, 3, 4)
True
\end{verbatim}



\section{Tuple assignment}
\label{tuple.assignment}
\index{tuple!assignment}
\index{assignment!tuple}
\index{swap pattern}
\index{pattern!swap}

It is often useful to swap the values of two variables.
With conventional assignments, you have to use a temporary
variable.  For example, to swap {\tt a} and {\tt b}:

\begin{verbatim}
>>> temp = a
>>> a = b
>>> b = temp
\end{verbatim}
%
This solution is cumbersome; {\bf tuple assignment} is more elegant:

\begin{verbatim}
>>> a, b = b, a
\end{verbatim}
%
The left side is a tuple of variables; the right side is a tuple of
expressions.  Each value is assigned to its respective variable.  
All the expressions on the right side are evaluated before any
of the assignments.

The number of variables on the left and the number of
values on the right have to be the same:
\index{exception!ValueError}
\index{ValueError}

\begin{verbatim}
>>> a, b = 1, 2, 3
ValueError: too many values to unpack
\end{verbatim}
%
More generally, the right side can be any kind of sequence
(string, list or tuple).  For example, to split an email address
into a user name and a domain, you could write:
\index{split method}
\index{method!split}
\index{email address}

\begin{verbatim}
>>> addr = 'monty@python.org'
>>> uname, domain = addr.split('@')
\end{verbatim}
%
The return value from {\tt split} is a list with two elements;
the first element is assigned to {\tt uname}, the second to
{\tt domain}.

\begin{verbatim}
>>> uname
'monty'
>>> domain
'python.org'
\end{verbatim}
%

\section{Tuples as return values}
\index{tuple}
\index{value!tuple}
\index{return value!tuple}
\index{function, tuple as return value}

Strictly speaking, a function can only return one value, but
if the value is a tuple, the effect is the same as returning
multiple values.  For example, if you want to divide two integers
and compute the quotient and remainder, it is inefficient to
compute {\tt x/y} and then {\tt x\%y}.  It is better to compute
them both at the same time.
\index{divmod}

The built-in function {\tt divmod} takes two arguments and
returns a tuple of two values, the quotient and remainder.
You can store the result as a tuple:

\begin{verbatim}
>>> t = divmod(7, 3)
>>> t
(2, 1)
\end{verbatim}
%
Or use tuple assignment to store the elements separately:
\index{tuple assignment}
\index{assignment!tuple}

\begin{verbatim}
>>> quot, rem = divmod(7, 3)
>>> quot
2
>>> rem
1
\end{verbatim}
%
Here is an example of a function that returns a tuple:

\begin{verbatim}
def min_max(t):
    return min(t), max(t)
\end{verbatim}
%
{\tt max} and {\tt min} are built-in functions that find
the largest and smallest elements of a sequence.  \verb"min_max"
computes both and returns a tuple of two values.
\index{max function}
\index{function!max}
\index{min function}
\index{function!min}


\section{Variable-length argument tuples}
\label{gather}
\index{variable-length argument tuple}
\index{argument!variable-length tuple}
\index{gather}
\index{parameter!gather}
\index{argument!gather}

Functions can take a variable number of arguments.  A parameter
name that begins with {\tt *} {\bf gathers} arguments into
a tuple.  For example, {\tt printall}
takes any number of arguments and prints them:

\begin{verbatim}
def printall(*args):
    print(args)
\end{verbatim}
%
The gather parameter can have any name you like, but {\tt args} is
conventional.  Here's how the function works:

\begin{verbatim}
>>> printall(1, 2.0, '3')
(1, 2.0, '3')
\end{verbatim}
%
The complement of gather is {\bf scatter}.  If you have a
sequence of values and you want to pass it to a function
as multiple arguments, you can use the {\tt *} operator.
For example, {\tt divmod} takes exactly two arguments; it
doesn't work with a tuple:
\index{scatter}
\index{argument scatter}
\index{TypeError}
\index{exception!TypeError}

\begin{verbatim}
>>> t = (7, 3)
>>> divmod(t)
TypeError: divmod expected 2 arguments, got 1
\end{verbatim}
%
But if you scatter the tuple, it works:

\begin{verbatim}
>>> divmod(*t)
(2, 1)
\end{verbatim}
%
Many of the built-in functions use
variable-length argument tuples.  For example, {\tt max}
and {\tt min} can take any number of arguments:
\index{max function}
\index{function!max}
\index{min function}
\index{function!min}

\begin{verbatim}
>>> max(1, 2, 3)
3
\end{verbatim}
%
But {\tt sum} does not.
\index{sum function}
\index{function!sum}

\begin{verbatim}
>>> sum(1, 2, 3)
TypeError: sum expected at most 2 arguments, got 3
\end{verbatim}
%
As an exercise, write a function called {\tt sumall} that takes any number
of arguments and returns their sum.


\section{Lists and tuples}
\index{zip function}
\index{function!zip}

{\tt zip} is a built-in function that takes two or more sequences and
returns a list of tuples where each tuple contains one
element from each sequence.  The name of the function refers to
a zipper, which joins and interleaves two rows of teeth.

This example zips a string and a list:

\begin{verbatim}
>>> s = 'abc'
>>> t = [0, 1, 2]
>>> zip(s, t)
<zip object at 0x7f7d0a9e7c48>
\end{verbatim}
%
The result is a {\bf zip object} that knows how to iterate through
the pairs.  The most common use of {\tt zip} is in a {\tt for} loop:

\begin{verbatim}
>>> for pair in zip(s, t):
...     print(pair)
...
('a', 0)
('b', 1)
('c', 2)
\end{verbatim}
%
A zip object is a kind of {\bf iterator}, which is any object
that iterates through a sequence.  Iterators are similar to lists in some
ways, but unlike lists, you can't use an index to select an element from
an iterator.
\index{iterator}

If you want to use list operators and methods, you can
use a zip object to make a list:

\begin{verbatim}
>>> list(zip(s, t))
[('a', 0), ('b', 1), ('c', 2)]
\end{verbatim}
%
The result is a list of tuples; in this example, each tuple contains
a character from the string and the corresponding element from
the list.
\index{list!of tuples}

If the sequences are not the same length, the result has the
length of the shorter one.

\begin{verbatim}
>>> list(zip('Anne', 'Elk'))
[('A', 'E'), ('n', 'l'), ('n', 'k')]
\end{verbatim}
%
You can use tuple assignment in a {\tt for} loop to traverse a list of
tuples:
\index{traversal}
\index{tuple assignment}
\index{assignment!tuple}

\begin{verbatim}
t = [('a', 0), ('b', 1), ('c', 2)]
for letter, number in t:
    print(number, letter)
\end{verbatim}
%
Each time through the loop, Python selects the next tuple in
the list and assigns the elements to {\tt letter} and 
{\tt number}.  The output of this loop is:
\index{loop}

\begin{verbatim}
0 a
1 b
2 c
\end{verbatim}
%
If you combine {\tt zip}, {\tt for} and tuple assignment, you get a
useful idiom for traversing two (or more) sequences at the same
time.  For example, \verb"has_match" takes two sequences, {\tt t1} and
{\tt t2}, and returns {\tt True} if there is an index {\tt i}
such that {\tt t1[i] == t2[i]}:
\index{for loop}

\begin{verbatim}
def has_match(t1, t2):
    for x, y in zip(t1, t2):
        if x == y:
            return True
    return False
\end{verbatim}
%
If you need to traverse the elements of a sequence and their
indices, you can use the built-in function {\tt enumerate}:
\index{traversal}
\index{enumerate function}
\index{function!enumerate}

\begin{verbatim}
for index, element in enumerate('abc'):
    print(index, element)
\end{verbatim}
%
The result from {\tt enumerate} is an enumerate object, which
iterates a sequence of pairs; each pair contains an index (starting
from 0) and an element from the given sequence.
In this example, the output is

\begin{verbatim}
0 a
1 b
2 c
\end{verbatim}
%
Again.
\index{iterator}
\index{object!enumerate}
\index{enumerate object}


\section{Dictionaries and tuples}
\label{dictuple}
\index{dictionary}
\index{items method}
\index{method!items}
\index{key-value pair}

Dictionaries have a method called {\tt items} that returns a sequence of
tuples, where each tuple is a key-value pair.

\begin{verbatim}
>>> d = {'a':0, 'b':1, 'c':2}
>>> t = d.items()
>>> t
dict_items([('c', 2), ('a', 0), ('b', 1)])
\end{verbatim}
%
The result is a \verb"dict_items" object, which is an iterator that
iterates the key-value pairs.  You can use it in a {\tt for} loop
like this:
\index{iterator}

\begin{verbatim}
>>> for key, value in d.items():
...     print(key, value)
...
c 2
a 0
b 1
\end{verbatim}
%
As you should expect from a dictionary, the items are in no
particular order.

Going in the other direction, you can use a list of tuples to
initialize a new dictionary: \index{dictionary!initialize}

\begin{verbatim}
>>> t = [('a', 0), ('c', 2), ('b', 1)]
>>> d = dict(t)
>>> d
{'a': 0, 'c': 2, 'b': 1}
\end{verbatim}

Combining {\tt dict} with {\tt zip} yields a concise way
to create a dictionary:
\index{zip function!use with dict}

\begin{verbatim}
>>> d = dict(zip('abc', range(3)))
>>> d
{'a': 0, 'c': 2, 'b': 1}
\end{verbatim}
%
The dictionary method {\tt update} also takes a list of tuples
and adds them, as key-value pairs, to an existing dictionary.
\index{update method}
\index{method!update}
\index{traverse!dictionary}
\index{dictionary!traversal}

It is common to use tuples as keys in dictionaries (primarily because
you can't use lists).  For example, a telephone directory might map
from last-name, first-name pairs to telephone numbers.  Assuming
that we have defined {\tt last}, {\tt first} and {\tt number}, we
could write:
\index{tuple!as key in dictionary}
\index{hashable}

\begin{verbatim}
directory[last, first] = number
\end{verbatim}
%
The expression in brackets is a tuple.  We could use tuple
assignment to traverse this dictionary.
\index{tuple!in brackets}

\begin{verbatim}
for last, first in directory:
    print(first, last, directory[last,first])
\end{verbatim}
%
This loop traverses the keys in {\tt directory}, which are tuples.  It
assigns the elements of each tuple to {\tt last} and {\tt first}, then
prints the name and corresponding telephone number.

There are two ways to represent tuples in a state diagram.  The more
detailed version shows the indices and elements just as they appear in
a list.  For example, the tuple \verb"('Cleese', 'John')" would appear
as in Figure~\ref{fig.tuple1}.
\index{state diagram}
\index{diagram!state}

\begin{figure}
\centerline
{\includegraphics[scale=0.8]{figs/tuple1.pdf}}
\caption{State diagram.}
\label{fig.tuple1}
\end{figure}

But in a larger diagram you might want to leave out the
details.  For example, a diagram of the telephone directory might
appear as in Figure~\ref{fig.dict2}.

\begin{figure}
\centerline
{\includegraphics[scale=0.8]{figs/dict2.pdf}}
\caption{State diagram.}
\label{fig.dict2}
\end{figure}

Here the tuples are shown using Python syntax as a graphical
shorthand.  The telephone number in the diagram is the complaints line
for the BBC, so please don't call it.


\section{Sequences of sequences}
\index{sequence}

I have focused on lists of tuples, but almost all of the examples in
this chapter also work with lists of lists, tuples of tuples, and
tuples of lists.  To avoid enumerating the possible combinations, it
is sometimes easier to talk about sequences of sequences.

In many contexts, the different kinds of sequences (strings, lists and
tuples) can be used interchangeably.  So how should you choose one
over the others?
\index{string}
\index{list}
\index{tuple}
\index{mutability}
\index{immutability}

To start with the obvious, strings are more limited than other
sequences because the elements have to be characters.  They are
also immutable.  If you need the ability to change the characters
in a string (as opposed to creating a new string), you might
want to use a list of characters instead.

Lists are more common than tuples, mostly because they are mutable.
But there are a few cases where you might prefer tuples:

\begin{enumerate}

\item In some contexts, like a {\tt return} statement, it is
syntactically simpler to create a tuple than a list.

\item If you want to use a sequence as a dictionary key, you
have to use an immutable type like a tuple or string.

\item If you are passing a sequence as an argument to a function,
using tuples reduces the potential for unexpected behavior
due to aliasing.

\end{enumerate}

Because tuples are immutable, they don't provide methods like {\tt
  sort} and {\tt reverse}, which modify existing lists.  But Python
provides the built-in function {\tt sorted}, which takes any sequence
and returns a new list with the same elements in sorted order, and
{\tt reversed}, which takes a sequence and returns an iterator that
traverses the list in reverse order.
\index{sorted function}
\index{function!sorted} \index{reversed function}
\index{function!reversed}
\index{iterator}


\section{Debugging}
\index{debugging}
\index{data structure}
\index{shape error}
\index{error!shape}

Lists, dictionaries and tuples are examples of {\bf data
  structures}; in this chapter we are starting to see compound data
structures, like lists of tuples, or dictionaries that contain tuples
as keys and lists as values.  Compound data structures are useful, but
they are prone to what I call {\bf shape errors}; that is, errors
caused when a data structure has the wrong type, size, or structure.
For example, if you are expecting a list with one integer and I
give you a plain old integer (not in a list), it won't work.
\index{structshape module}
\index{module!structshape}

To help debug these kinds of errors, I have written a module
called {\tt structshape} that provides a function, also called
{\tt structshape}, that takes any kind of data structure as
an argument and returns a string that summarizes its shape.
You can download it from \url{http://thinkpython2.com/code/structshape.py}

Here's the result for a simple list:

\begin{verbatim}
>>> from structshape import structshape
>>> t = [1, 2, 3]
>>> structshape(t)
'list of 3 int'
\end{verbatim}
%
A fancier program might write ``list of 3 int{\em s}'', but it
was easier not to deal with plurals.  Here's a list of lists:

\begin{verbatim}
>>> t2 = [[1,2], [3,4], [5,6]]
>>> structshape(t2)
'list of 3 list of 2 int'
\end{verbatim}
%
If the elements of the list are not the same type,
{\tt structshape} groups them, in order, by type:

\begin{verbatim}
>>> t3 = [1, 2, 3, 4.0, '5', '6', [7], [8], 9]
>>> structshape(t3)
'list of (3 int, float, 2 str, 2 list of int, int)'
\end{verbatim}
%
Here's a list of tuples:

\begin{verbatim}
>>> s = 'abc'
>>> lt = list(zip(t, s))
>>> structshape(lt)
'list of 3 tuple of (int, str)'
\end{verbatim}
%
And here's a dictionary with 3 items that map integers to strings.

\begin{verbatim}
>>> d = dict(lt) 
>>> structshape(d)
'dict of 3 int->str'
\end{verbatim}
%
If you are having trouble keeping track of your data structures,
{\tt structshape} can help.


\section{Glossary}

\begin{description}

\item[tuple:] An immutable sequence of elements.
\index{tuple}

\item[tuple assignment:] An assignment with a sequence on the
right side and a tuple of variables on the left.  The right
side is evaluated and then its elements are assigned to the
variables on the left.
\index{tuple assignment}
\index{assignment!tuple}

\item[gather:] The operation of assembling a variable-length
argument tuple.
\index{gather}

\item[scatter:] The operation of treating a sequence as a list of
arguments.
\index{scatter}

\item[zip object:] The result of calling a built-in function {\tt zip};
an object that iterates through a sequence of tuples.
\index{zip object}
\index{object!zip}

\item[iterator:] An object that can iterate through a sequence, but
which does not provide list operators and methods.
\index{iterator}

\item[data structure:] A collection of related values, often
organized in lists, dictionaries, tuples, etc.
\index{data structure}

\item[shape error:] An error caused because a value has the
wrong shape; that is, the wrong type or size.
\index{shape}

\end{description}


\section{Exercises}

\begin{exercise}

Write a function called \verb"most_frequent" that takes a string and
prints the letters in decreasing order of frequency.  Find text
samples from several different languages and see how letter frequency
varies between languages.  Compare your results with the tables at
\url{http://en.wikipedia.org/wiki/Letter_frequencies}.  Solution:
\url{http://thinkpython2.com/code/most_frequent.py}.  \index{letter
  frequency} \index{frequency!letter}

\end{exercise}


\begin{exercise}
\label{anagrams}
\index{anagram set}
\index{set!anagram}

More anagrams!

\begin{enumerate}

\item Write a program
that reads a word list from a file (see Section~\ref{wordlist}) and
prints all the sets of words that are anagrams.

Here is an example of what the output might look like:

\begin{verbatim}
['deltas', 'desalt', 'lasted', 'salted', 'slated', 'staled']
['retainers', 'ternaries']
['generating', 'greatening']
['resmelts', 'smelters', 'termless']
\end{verbatim}
%
Hint: you might want to build a dictionary that maps from a
collection of letters to a list of words that can be spelled with those
letters.  The question is, how can you represent the collection of
letters in a way that can be used as a key?

\item Modify the previous program so that it prints the longest list
of anagrams first, followed by the second longest, and so on.
\index{Scrabble}
\index{bingo}

\item In Scrabble a ``bingo'' is when you play all seven tiles in
your rack, along with a letter on the board, to form an eight-letter
word.  What collection of 8 letters forms the most possible bingos?
Hint: there are seven.

% (7, ['angriest', 'astringe', 'ganister', 'gantries', 'granites',
% 'ingrates', 'rangiest'])

Solution: \url{http://thinkpython2.com/code/anagram_sets.py}.

\end{enumerate}
\end{exercise}

\begin{exercise}
\index{metathesis}

Two words form a ``metathesis pair'' if you can transform one into the
other by swapping two letters; for example, ``converse'' and
``conserve''.  Write a program that finds all of the metathesis pairs
in the dictionary.  Hint: don't test all pairs of words, and don't
test all possible swaps.  Solution:
\url{http://thinkpython2.com/code/metathesis.py}.  Credit: This
exercise is inspired by an example at \url{http://puzzlers.org}.

\end{exercise}


\begin{exercise}
\index{Car Talk}
\index{Puzzler}

Here's another Car Talk Puzzler
(\url{http://www.cartalk.com/content/puzzlers}):

\begin{quote}
What is the longest English word, that remains a valid English word,
as you remove its letters one at a time?

Now, letters can be removed from either end, or the middle, but you
can't rearrange any of the letters. Every time you drop a letter, you
wind up with another English word. If you do that, you're eventually
going to wind up with one letter and that too is going to be an
English word---one that's found in the dictionary. I want to know
what's the longest word and how many letters does it
have?

I'm going to give you a little modest example: Sprite. Ok? You start
off with sprite, you take a letter off, one from the interior of the
word, take the r away, and we're left with the word spite, then we
take the e off the end, we're left with spit, we take the s off, we're
left with pit, it, and I.
\end{quote}
\index{reducible word}
\index{word, reducible}

Write a program to find all words that can be reduced in this way,
and then find the longest one.

This exercise is a little more challenging than most, so here are
some suggestions:

\begin{enumerate}

\item You might want to write a function that takes a word and
  computes a list of all the words that can be formed by removing one
  letter.  These are the ``children'' of the word.
\index{recursive definition}
\index{definition!recursive}

\item Recursively, a word is reducible if any of its children
are reducible.  As a base case, you can consider the empty
string reducible.

\item The wordlist I provided, {\tt words.txt}, doesn't
contain single letter words.  So you might want to add
``I'', ``a'', and the empty string.

\item To improve the performance of your program, you might want
to memoize the words that are known to be reducible.

\end{enumerate}

Solution: \url{http://thinkpython2.com/code/reducible.py}.

\end{exercise}

%\begin{exercise}
%\url{http://en.wikipedia.org/wiki/Word_Ladder}
%\end{exercise}
\chapter{Case study: interface design}
\label{turtlechap}

This chapter presents a case study that demonstrates a process for
designing functions that work together.

It introduces the {\tt turtle} module, which allows you to
create images using turtle graphics.  The {\tt turtle} module is
included in most Python installations, but if you are running Python
using PythonAnywhere, you won't be able to run the turtle examples (at
least you couldn't when I wrote this).

If you have already installed Python on your computer, you should
be able to run the examples.  Otherwise, now is a good time
to install.  I have posted instructions at
\url{http://tinyurl.com/thinkpython2e}.

Code examples from this chapter are available from
\url{http://thinkpython2.com/code/polygon.py}.


\section{The turtle module}
\label{turtle}

To check whether you have the {\tt turtle} module, open the Python
interpreter and type

\begin{verbatim}
>>> import turtle
>>> bob = turtle.Turtle()
\end{verbatim}

When you run this code, it should create a new window
with small arrow that represents the turtle.  Close the window.

Create a file named {\tt mypolygon.py} and type in the following
code:

\begin{verbatim}
import turtle
bob = turtle.Turtle()
print(bob)
turtle.mainloop()
\end{verbatim}
%
The {\tt turtle} module (with a lowercase 't') provides a function
called {\tt Turtle} (with an uppercase 'T') that creates a Turtle
object, which we assign to a variable named {\tt bob}.
Printing {\tt bob} displays something like:

\begin{verbatim}
<turtle.Turtle object at 0xb7bfbf4c>
\end{verbatim}
%
This means that {\tt bob} refers to an object with type
{\tt Turtle}
as defined in module {\tt turtle}.

\verb"mainloop" tells the window to wait for the user
to do something, although in this case there's not much for
the user to do except close the window.

Once you create a Turtle, you can call a {\bf method} to move it
around the window.  A method is similar to a function, but it
uses slightly different syntax.  For example, to move the turtle
forward:

\begin{verbatim}
bob.fd(100)
\end{verbatim}
%
The method, {\tt fd}, is associated with the turtle
object we're calling {\tt bob}.  
Calling a method is like making a request: you are asking {\tt bob}
to move forward.

The argument of {\tt fd} is a distance in pixels, so the actual
size depends on your display.

Other methods you can call on a Turtle are {\tt bk} to move
backward, {\tt lt} for left turn, and {\tt rt} right turn.  The
argument for {\tt lt} and {\tt rt} is an angle in degrees.

Also, each Turtle is holding a pen, which is
either down or up; if the pen is down, the Turtle leaves
a trail when it moves.  The methods {\tt pu} and {\tt pd}
stand for ``pen up'' and ``pen down''.

To draw a right angle, add these lines to the program
(after creating {\tt bob} and before calling \verb"mainloop"):

\begin{verbatim}
bob.fd(100)
bob.lt(90)
bob.fd(100)
\end{verbatim}
%
When you run this program, you should see {\tt bob} move east and then
north, leaving two line segments behind.

Now modify the program to draw a square.  Don't go on until
you've got it working!

%\newpage

\section{Simple repetition}
\label{repetition}
\index{repetition}

Chances are you wrote something like this:

\begin{verbatim}
bob.fd(100)
bob.lt(90)

bob.fd(100)
bob.lt(90)

bob.fd(100)
bob.lt(90)

bob.fd(100)
\end{verbatim}
%
We can do the same thing more concisely with a {\tt for} statement.
Add this example to {\tt mypolygon.py} and run it again:
\index{for loop}
\index{loop!for}
\index{statement!for}

\begin{verbatim}
for i in range(4):
    print('Hello!')
\end{verbatim}
%
You should see something like this:

\begin{verbatim}
Hello!
Hello!
Hello!
Hello!
\end{verbatim}
%
This is the simplest use of the {\tt for} statement; we will see
more later.  But that should be enough to let you rewrite your
square-drawing program.  Don't go on until you do.

Here is a {\tt for} statement that draws a square:

\begin{verbatim}
for i in range(4):
    bob.fd(100)
    bob.lt(90)
\end{verbatim}
%
The syntax of a {\tt for} statement is similar to a function
definition.  It has a header that ends with a colon and an indented
body.  The body can contain any number of statements.

A {\tt for} statement is also called a {\bf loop} because
the flow of execution runs through the body and then loops back
to the top.  In this case, it runs the body four times.
\index{loop}

This version is actually a little different from the previous
square-drawing code because it makes another turn after
drawing the last side of the square.  The extra turn takes
more time, but it simplifies the code if we do the same thing
every time through the loop.  This version also has the effect
of leaving the turtle back in the starting position, facing in
the starting direction.

\section{Exercises}

The following is a series of exercises using TurtleWorld.  They
are meant to be fun, but they have a point, too.  While you are
working on them, think about what the point is.

The following sections have solutions to the exercises, so
don't look until you have finished (or at least tried).

\begin{enumerate}

\item Write a function called {\tt square} that takes a parameter
named {\tt t}, which is a turtle.  It should use the turtle to draw
a square.

Write a function call that passes {\tt bob} as an argument to
{\tt square}, and then run the program again.

\item Add another parameter, named {\tt length}, to {\tt square}.
Modify the body so length of the sides is {\tt length}, and then
modify the function call to provide a second argument.  Run the
program again.  Test your program with a range of values for {\tt
length}.

\item Make a copy of {\tt square} and change the name to {\tt
  polygon}.  Add another parameter named {\tt n} and modify the body
  so it draws an n-sided regular polygon.  Hint: The exterior angles
  of an n-sided regular polygon are $360/n$ degrees.  \index{polygon
    function} \index{function!polygon}

\item Write a function called {\tt circle} that takes a turtle, 
{\tt t}, and radius, {\tt r}, as parameters and that draws an
approximate circle by calling {\tt polygon} with an appropriate
length and number of sides.  Test your function with a range of values
of {\tt r}.  \index{circle function} \index{function!circle}

Hint: figure out the circumference of the circle and make sure that
{\tt length * n = circumference}.

\item Make a more general version of {\tt circle} called {\tt arc}
that takes an additional parameter {\tt angle}, which determines
what fraction of a circle to draw.  {\tt angle} is in units of
degrees, so when {\tt angle=360}, {\tt arc} should draw a complete
circle.
\index{arc function}
\index{function!arc}

\end{enumerate}


\section{Encapsulation}

The first exercise asks you to put your square-drawing code
into a function definition and then call the function, passing
the turtle as a parameter.  Here is a solution:

\begin{verbatim}
def square(t):
    for i in range(4):
        t.fd(100)
        t.lt(90)

square(bob)
\end{verbatim}
%
The innermost statements, {\tt fd} and {\tt lt} are indented twice to
show that they are inside the {\tt for} loop, which is inside the
function definition.  The next line, {\tt square(bob)}, is flush with
the left margin, which indicates the end of both the {\tt for} loop
and the function definition.

Inside the function, {\tt t} refers to the same turtle {\tt bob}, so
{\tt t.lt(90)} has the same effect as {\tt bob.lt(90)}.  In that
case, why not
call the parameter {\tt bob}?  The idea is that {\tt t} can be any
turtle, not just {\tt bob}, so you could create a second turtle and
pass it as an argument to {\tt square}:

\begin{verbatim}
alice = turtle.Turtle()
square(alice)
\end{verbatim}
%
Wrapping a piece of code up in a function is called {\bf
encapsulation}.  One of the benefits of encapsulation is that it
attaches a name to the code, which serves as a kind of documentation.
Another advantage is that if you re-use the code, it is more concise
to call a function twice than to copy and paste the body!
\index{encapsulation}


\section{Generalization}

The next step is to add a {\tt length} parameter to {\tt square}.
Here is a solution:

\begin{verbatim}
def square(t, length):
    for i in range(4):
        t.fd(length)
        t.lt(90)

square(bob, 100)
\end{verbatim}
%
Adding a parameter to a function is called {\bf generalization}
because it makes the function more general: in the previous
version, the square is always the same size; in this version
it can be any size.
\index{generalization}

The next step is also a generalization.  Instead of drawing
squares, {\tt polygon} draws regular polygons with any number of
sides.  Here is a solution:

\begin{verbatim}
def polygon(t, n, length):
    angle = 360 / n
    for i in range(n):
        t.fd(length)
        t.lt(angle)

polygon(bob, 7, 70)
\end{verbatim}
%
This example draws a 7-sided polygon with side length 70.

If you are using Python 2, the value of {\tt angle} might be off
because of integer division.  A simple solution is to compute
{\tt angle = 360.0 / n}.  Because the numerator is a floating-point
number, the result is floating point.
\index{Python 2}

When a function has more than a few numeric arguments, it is easy to
forget what they are, or what order they should be in.  In that case
it is often a good idea to include the names of the parameters in the
argument list:

\begin{verbatim}
polygon(bob, n=7, length=70)
\end{verbatim}
%
These are called {\bf keyword arguments} because they include
the parameter names as ``keywords'' (not to be confused with
Python keywords like {\tt while} and {\tt def}).
\index{keyword argument}
\index{argument!keyword}

This syntax makes the program more readable.  It is also a reminder
about how arguments and parameters work: when you call a function, the
arguments are assigned to the parameters.


\section{Interface design}

The next step is to write {\tt circle}, which takes a radius,
{\tt r}, as a parameter.  Here is a simple solution that uses
{\tt polygon} to draw a 50-sided polygon:

\begin{verbatim}
import math

def circle(t, r):
    circumference = 2 * math.pi * r
    n = 50
    length = circumference / n
    polygon(t, n, length)
\end{verbatim}
%
The first line computes the circumference of a circle with radius
{\tt r} using the formula $2 \pi r$.  Since we use {\tt math.pi}, we
have to import {\tt math}.  By convention, {\tt import} statements
are usually at the beginning of the script.

{\tt n} is the number of line segments in our approximation of a circle,
so {\tt length} is the length of each segment.  Thus, {\tt polygon}
draws a 50-sided polygon that approximates a circle with radius {\tt r}.

One limitation of this solution is that {\tt n} is a constant, which
means that for very big circles, the line segments are too long, and
for small circles, we waste time drawing very small segments.  One
solution would be to generalize the function by taking {\tt n} as
a parameter.  This would give the user (whoever calls {\tt circle})
more control, but the interface would be less clean.
\index{interface}

The {\bf interface} of a function is a summary of how it is used: what
are the parameters?  What does the function do?  And what is the return
value?  An interface is ``clean'' if it allows the caller to do
what they want without dealing with unnecessary details.

In this example, {\tt r} belongs in the interface because it
specifies the circle to be drawn.  {\tt n} is less appropriate
because it pertains to the details of {\em how} the circle should
be rendered.

Rather than clutter up the interface, it is better
to choose an appropriate value of {\tt n}
depending on {\tt circumference}:

\begin{verbatim}
def circle(t, r):
    circumference = 2 * math.pi * r
    n = int(circumference / 3) + 3
    length = circumference / n
    polygon(t, n, length)
\end{verbatim}
%
Now the number of segments is an integer near {\tt circumference/3},
so the length of each segment is approximately 3, which is small
enough that the circles look good, but big enough to be efficient,
and acceptable for any size circle.

Adding 3 to {\tt n} guarantees that the polygon has at least 3 sides.


\section{Refactoring}
\label{refactoring}
\index{refactoring}

When I wrote {\tt circle}, I was able to re-use {\tt polygon}
because a many-sided polygon is a good approximation of a circle.
But {\tt arc} is not as cooperative; we can't use {\tt polygon}
or {\tt circle} to draw an arc.

One alternative is to start with a copy
of {\tt polygon} and transform it into {\tt arc}.  The result
might look like this:

\begin{verbatim}
def arc(t, r, angle):
    arc_length = 2 * math.pi * r * angle / 360
    n = int(arc_length / 3) + 1
    step_length = arc_length / n
    step_angle = angle / n
    
    for i in range(n):
        t.fd(step_length)
        t.lt(step_angle)
\end{verbatim}
%
The second half of this function looks like {\tt polygon}, but we
can't re-use {\tt polygon} without changing the interface.  We could
generalize {\tt polygon} to take an angle as a third argument,
but then {\tt polygon} would no longer be an appropriate name!
Instead, let's call the more general function {\tt polyline}:

\begin{verbatim}
def polyline(t, n, length, angle):
    for i in range(n):
        t.fd(length)
        t.lt(angle)
\end{verbatim}
%
Now we can rewrite {\tt polygon} and {\tt arc} to use {\tt polyline}:

\begin{verbatim}
def polygon(t, n, length):
    angle = 360.0 / n
    polyline(t, n, length, angle)

def arc(t, r, angle):
    arc_length = 2 * math.pi * r * angle / 360
    n = int(arc_length / 3) + 1
    step_length = arc_length / n
    step_angle = float(angle) / n
    polyline(t, n, step_length, step_angle)
\end{verbatim}
%
Finally, we can rewrite {\tt circle} to use {\tt arc}:

\begin{verbatim}
def circle(t, r):
    arc(t, r, 360)
\end{verbatim}
%
This process---rearranging a program to improve
interfaces and facilitate code re-use---is called {\bf refactoring}.
In this case, we noticed that there was similar code in {\tt arc} and
{\tt polygon}, so we ``factored it out'' into {\tt polyline}.
\index{refactoring}

If we had planned ahead, we might have written {\tt polyline} first
and avoided refactoring, but often you don't know enough at the
beginning of a project to design all the interfaces.  Once you start
coding, you understand the problem better.  Sometimes refactoring is a
sign that you have learned something.


\section{A development plan}
\index{development plan!encapsulation and generalization}

A {\bf development plan} is a process for writing programs.  The
process we used in this case study is ``encapsulation and
generalization''.  The steps of this process are:

\begin{enumerate}

\item Start by writing a small program with no function definitions.

\item Once you get the program working, identify a coherent piece of
  it, encapsulate the piece in a function and give it a name.

\item Generalize the function by adding appropriate parameters.

\item Repeat steps 1--3 until you have a set of working functions.
Copy and paste working code to avoid retyping (and re-debugging).

\item Look for opportunities to improve the program by refactoring.
For example, if you have similar code in several places, consider
factoring it into an appropriately general function.

\end{enumerate}

This process has some drawbacks---we will see alternatives later---but
it can be useful if you don't know ahead of time how to divide the
program into functions.  This approach lets you design as you go
along.


\section{docstring}
\label{docstring}
\index{docstring}

A {\bf docstring} is a string at the beginning of a function that
explains the interface (``doc'' is short for ``documentation'').  Here
is an example:

\begin{verbatim}
def polyline(t, n, length, angle):
    """Draws n line segments with the given length and
    angle (in degrees) between them.  t is a turtle.
    """    
    for i in range(n):
        t.fd(length)
        t.lt(angle)
\end{verbatim}
%
By convention, all docstrings are triple-quoted strings, also known
as multiline strings because the triple quotes allow the string
to span more than one line.
\index{quotation mark}
\index{triple-quoted string}
\index{string!triple-quoted}
\index{multiline string}
\index{string!multiline}

It is terse, but it contains the essential information
someone would need to use this function.  It explains concisely what
the function does (without getting into the details of how it does
it).  It explains what effect each parameter has on the behavior of
the function and what type each parameter should be (if it is not
obvious).

Writing this kind of documentation is an important part of interface
design.  A well-designed interface should be simple to explain;
if you have a hard time explaining one of your functions,
maybe the interface could be improved.


\section{Debugging}
\index{debugging}
\index{interface}

An interface is like a contract between a function and a caller.
The caller agrees to provide certain parameters and the function
agrees to do certain work.

For example, {\tt polyline} requires four arguments: {\tt t} has to be
a Turtle; {\tt n} has to be an
integer; {\tt length} should be a positive number; and {\tt
  angle} has to be a number, which is understood to be in degrees.

These requirements are called {\bf preconditions} because they
are supposed to be true before the function starts executing.
Conversely, conditions at the end of the function are
{\bf postconditions}.  Postconditions include the intended
effect of the function (like drawing line segments) and any
side effects (like moving the Turtle or making other changes).
\index{precondition}
\index{postcondition}

Preconditions are the responsibility of the caller.  If the caller
violates a (properly documented!) precondition and the function
doesn't work correctly, the bug is in the caller, not the function.

If the preconditions are satisfied and the postconditions are
not, the bug is in the function.  If your pre- and postconditions
are clear, they can help with debugging.


\section{Glossary}

\begin{description}

\item[method:] A function that is associated with an object and called
using dot notation.
\index{method}

\item[loop:] A part of a program that can run repeatedly.
\index{loop}

\item[encapsulation:] The process of transforming a sequence of
statements into a function definition.
\index{encapsulation}

\item[generalization:] The process of replacing something
unnecessarily specific (like a number) with something appropriately
general (like a variable or parameter).
\index{generalization}

\item[keyword argument:] An argument that includes the name of
the parameter as a ``keyword''.
\index{keyword argument}
\index{argument!keyword}

\item[interface:] A description of how to use a function, including
the name and descriptions of the arguments and return value.
\index{interface}

\item[refactoring:] The process of modifying a working program to
  improve function interfaces and other qualities of the code.
\index{refactoring}

\item[development plan:] A process for writing programs.
\index{development plan}

\item[docstring:] A string that appears at the top of a function
  definition to document the function's interface. 
\index{docstring}

\item[precondition:] A requirement that should be satisfied by
the caller before a function starts.
\index{precondition}

\item[postcondition:] A requirement that should be satisfied by
the function before it ends.
\index{precondition}

\end{description}


\section{Exercises}

\begin{exercise}

Download the code in this chapter from
\url{http://thinkpython2.com/code/polygon.py}.

\begin{enumerate}

\item Draw a stack diagram that shows the state of the program
while executing {\tt circle(bob, radius)}.  You can do the
arithmetic by hand or add {\tt print} statements to the code.
\index{stack diagram}

\item The version of {\tt arc} in Section~\ref{refactoring} is not
very accurate because the linear approximation of the
circle is always outside the true circle.  As a result,
the Turtle ends up a few pixels away from the correct
destination.  My solution shows a way to reduce
the effect of this error.  Read the code and see if it makes
sense to you.  If you draw a diagram, you might see how it works.

\end{enumerate}

\end{exercise}

\begin{figure}
\centerline
{\includegraphics[scale=0.8]{figs/flowers.pdf}}
\caption{Turtle flowers.}
\label{fig.flowers}
\end{figure}

\begin{exercise}
\index{flower}

Write an appropriately general set of functions that
can draw flowers as in Figure~\ref{fig.flowers}.

Solution: \url{http://thinkpython2.com/code/flower.py},
also requires \url{http://thinkpython2.com/code/polygon.py}.

\end{exercise}

\begin{figure}
\centerline
{\includegraphics[scale=0.8]{figs/pies.pdf}}
\caption{Turtle pies.}
\label{fig.pies}
\end{figure}


\begin{exercise}
\index{pie}

Write an appropriately general set of functions that
can draw shapes as in Figure~\ref{fig.pies}.

Solution: \url{http://thinkpython2.com/code/pie.py}.

\end{exercise}

\begin{exercise}
\index{alphabet}
\index{turtle typewriter}
\index{typewriter, turtle}

The letters of the alphabet can be constructed from a moderate number
of basic elements, like vertical and horizontal lines and a few
curves.  Design an alphabet that can be drawn with a minimal
number of basic elements and then write functions that draw the letters.

You should write one function for each letter, with names
\verb"draw_a", \verb"draw_b", etc., and put your functions
in a file named {\tt letters.py}.  You can download a
``turtle typewriter'' from \url{http://thinkpython2.com/code/typewriter.py}
to help you test your code.

You can get a solution from \url{http://thinkpython2.com/code/letters.py};
it also requires
\url{http://thinkpython2.com/code/polygon.py}.

\end{exercise}

\begin{exercise}

Read about spirals at \url{http://en.wikipedia.org/wiki/Spiral}; then
write a program that draws an Archimedian spiral (or one of the other
kinds).  Solution: \url{http://thinkpython2.com/code/spiral.py}.
\index{spiral}
\index{Archimedian spiral} 

\end{exercise}
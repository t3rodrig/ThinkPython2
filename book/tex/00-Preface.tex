\chapter{Preface}

\section*{The strange history of this book}

In January 1999 I was preparing to teach an introductory programming
class in Java.  I had taught it three times and I was getting
frustrated.  The failure rate in the class was too high and, even for
students who succeeded, the overall level of achievement was too low.

One of the problems I saw was the books.  
They were too big, with too much unnecessary detail about Java, and
not enough high-level guidance about how to program.  And they all
suffered from the trap door effect: they would start out easy,
proceed gradually, and then somewhere around Chapter 5 the bottom would
fall out.  The students would get too much new material, too fast,
and I would spend the rest of the semester picking up the pieces.

Two weeks before the first day of classes, I decided to write my
own book.  My goals were:

\begin{itemize}

\item Keep it short.  It is better for students to read 10 pages
than not read 50 pages.

\item Be careful with vocabulary.  I tried to minimize jargon
and define each term at first use.

\item Build gradually. To avoid trap doors, I took the most difficult
topics and split them into a series of small steps. 

\item Focus on programming, not the programming language.  I included
the minimum useful subset of Java and left out the rest.

\end{itemize}

I needed a title, so on a whim I chose {\em How to Think Like
a Computer Scientist}.

My first version was rough, but it worked.  Students did the reading,
and they understood enough that I could spend class time on the hard
topics, the interesting topics and (most important) letting the
students practice.

I released the book under the GNU Free Documentation License,
which allows users to copy, modify, and distribute the book.
\index{GNU Free Documentation License}
\index{Free Documentation License, GNU}

What happened next is the cool part.  Jeff Elkner, a high school
teacher in Virginia, adopted my book and translated it into
Python.  He sent me a copy of his translation, and I had the
unusual experience of learning Python by reading my own book.
As Green Tea Press, I published the first Python version in 2001.
\index{Elkner, Jeff}

In 2003 I started teaching at Olin College and I got to teach
Python for the first time.  The contrast with Java was striking.
Students struggled less, learned more, worked on more interesting
projects, and generally had a lot more fun.
\index{Olin College}

Since then I've continued to develop the book,
correcting errors, improving some of the examples and
adding material, especially exercises.

The result is this book, now with the less grandiose title
{\em Think Python}.  Some of the changes are:

\begin{itemize}

\item I added a section about debugging at the end of each chapter.
  These sections present general techniques for finding and avoiding
  bugs, and warnings about Python pitfalls.

\item I added more exercises, ranging from short tests of
  understanding to a few substantial projects.  Most exercises
  include a link to my solution.

\item I added a series of case studies---longer examples with
  exercises, solutions, and discussion.
  
\item I expanded the discussion of program development plans
  and basic design patterns.

\item I added appendices about debugging and analysis of algorithms.

\end{itemize}

The second edition of {\em Think Python} has these new features:

\begin{itemize}

\item The book and all supporting code have been updated to Python 3.

\item I added a few sections, and more details on the web, to help
beginners get started running Python in a browser, so you don't have
to deal with installing Python until you want to.

\item For Chapter~\ref{turtle} I switched from my own turtle graphics
  package, called Swampy, to a more standard Python module, {\tt
    turtle}, which is easier to install and more powerful.

\item I added a new chapter called ``The Goodies'', which introduces
some additional Python features that are not strictly necessary, but
sometimes handy.

\end{itemize}

I hope you enjoy working with this book, and that it helps
you learn to program and think like
a computer scientist, at least a little bit.


Allen B. Downey \\

Olin College \\


\section*{Acknowledgments}

Many thanks to Jeff Elkner, who
translated my Java book into Python, which got this project
started and introduced me to what has turned out to be my
favorite language.
\index{Elkner, Jeff}

Thanks also to Chris Meyers, who contributed several sections
to {\em How to Think Like a Computer Scientist}.
\index{Meyers, Chris}

Thanks to the Free Software Foundation for developing
the GNU Free Documentation License, which helped make
my collaboration with Jeff and Chris possible, and Creative
Commons for the license I am using now.
\index{GNU Free Documentation License}
\index{Free Documentation License, GNU}
\index{Creative Commons}

Thanks to the editors at Lulu who worked on
{\em How to Think Like a Computer Scientist}.

Thanks to the editors at O'Reilly Media who worked on
{\em Think Python}.

Thanks to all the students who worked with earlier
versions of this book and all the contributors (listed
below) who sent in corrections and suggestions.

\input{tex/Contributor-List}
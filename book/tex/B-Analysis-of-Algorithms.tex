\chapter{Analysis of Algorithms}
\label{algorithms}

\begin{quote}
This appendix is an edited excerpt from {\it Think Complexity}, by
Allen B. Downey, also published by O'Reilly Media (2012).  When you
are done with this book, you might want to move on to that one.
\end{quote}

{\bf Analysis of algorithms} is a branch of computer science that
studies the performance of algorithms, especially their run time and
space requirements.  See
\url{http://en.wikipedia.org/wiki/Analysis_of_algorithms}.
\index{algorithm} \index{analysis of algorithms}

The practical goal of algorithm analysis is to predict the performance
of different algorithms in order to guide design decisions.

During the 2008 United States Presidential Campaign, candidate
Barack Obama was asked to perform an impromptu analysis when
he visited Google.  Chief executive Eric Schmidt jokingly asked him
for ``the most efficient way to sort a million 32-bit integers.''
Obama had apparently been tipped off, because he quickly
replied, ``I think the bubble sort would be the wrong way to go.''
See \url{http://www.youtube.com/watch?v=k4RRi_ntQc8}.
\index{Obama, Barack}
\index{Schmidt, Eric}
\index{bubble sort}

This is true: bubble sort is conceptually simple but slow for
large datasets.  The answer Schmidt was probably looking for is
``radix sort'' (\url{http://en.wikipedia.org/wiki/Radix_sort})\footnote{
But if you get a question like this in an interview, I think
a better answer is, ``The fastest way to sort a million integers
is to use whatever sort function is provided by the language
I'm using.  Its performance is good enough for the vast majority
of applications, but if it turned out that my application was too
slow, I would use a profiler to see where the time was being
spent.  If it looked like a faster sort algorithm would have
a significant effect on performance, then I would look
around for a good implementation of radix sort.''}.
\index{radix sort}

The goal of algorithm analysis is to make meaningful
comparisons between algorithms, but there are some problems:
\index{comparing algorithms}

\begin{itemize}

\item The relative performance of the algorithms might
depend on characteristics of the hardware, so one algorithm
might be faster on Machine A, another on Machine B.
The general solution to this problem is to specify a
{\bf machine model} and analyze the number of steps, or
operations, an algorithm requires under a given model.
\index{machine model}

\item Relative performance might depend on the details of
the dataset.  For example, some sorting
algorithms run faster if the data are already partially sorted;
other algorithms run slower in this case.
A common way to avoid this problem is to analyze the
{\bf worst case} scenario.  It is sometimes useful to
analyze average case performance, but that's usually harder,
and it might not be obvious what set of cases to average over.
\index{worst case}
\index{average case}

\item Relative performance also depends on the size of the
problem.  A sorting algorithm that is fast for small lists
might be slow for long lists.
The usual solution to this problem is to express run time
(or number of operations) as a function of problem size,
and group functions into categories depending on how quickly
they grow as problem size increases.

\end{itemize}

The good thing about this kind of comparison is that it lends
itself to simple classification of algorithms.  For example,
if I know that the run time of Algorithm A tends to be
proportional to the size of the input, $n$, and Algorithm B
tends to be proportional to $n^2$, then I
expect A to be faster than B, at least for large values of $n$.

This kind of analysis comes with some caveats, but we'll get
to that later.


\section{Order of growth}

Suppose you have analyzed two algorithms and expressed
their run times in terms of the size of the input:
Algorithm A takes $100n+1$ steps to solve a problem with
size $n$; Algorithm B takes $n^2 + n + 1$ steps.
\index{order of growth}

The following table shows the run time of these algorithms
for different problem sizes:

\begin{tabular}{|r|r|r|}
\hline
Input     &   Run time of     & Run time of \\
size      &   Algorithm A     & Algorithm B \\
\hline
10        &   1 001           & 111         \\
100       &   10 001          & 10 101         \\
1 000     &   100 001         & 1 001 001         \\
10 000    &   1 000 001       & $> 10^{10}$         \\
\hline
\end{tabular}

At $n=10$, Algorithm A looks pretty bad; it takes almost 10 times
longer than Algorithm B.  But for $n=100$ they are about the same, and
for larger values A is much better.

The fundamental reason is that for large values of $n$, any function
that contains an $n^2$ term will grow faster than a function whose
leading term is $n$.  The {\bf leading term} is the term with the
highest exponent.
\index{leading term}
\index{exponent}

For Algorithm A, the leading term has a large coefficient, 100, which
is why B does better than A for small $n$.  But regardless of the
coefficients, there will always be some value of $n$ where
$a n^2 > b n$, for any values of $a$ and $b$.
\index{leading coefficient}

The same argument applies to the non-leading terms.  Even if the run
time of Algorithm A were $n+1000000$, it would still be better than
Algorithm B for sufficiently large $n$.

In general, we expect an algorithm with a smaller leading term to be a
better algorithm for large problems, but for smaller problems, there
may be a {\bf crossover point} where another algorithm is better.  The
location of the crossover point depends on the details of the
algorithms, the inputs, and the hardware, so it is usually ignored for
purposes of algorithmic analysis.  But that doesn't mean you can forget
about it.
\index{crossover point}

If two algorithms have the same leading order term, it is hard to say
which is better; again, the answer depends on the details.  So for
algorithmic analysis, functions with the same leading term
are considered equivalent, even if they have different coefficients.

An {\bf order of growth} is a set of functions whose growth
behavior is considered equivalent.  For example, $2n$, $100n$ and $n+1$ 
belong to the same order of growth, which is written $O(n)$ in
{\bf Big-Oh notation} and often called {\bf linear} because every function
in the set grows linearly with $n$.
\index{big-oh notation}
\index{linear growth}

All functions with the leading term $n^2$ belong to $O(n^2)$; they are
called {\bf quadratic}.
\index{quadratic growth}

The following table shows some of the orders of growth that
appear most commonly in algorithmic analysis,
in increasing order of badness.
\index{badness}

\begin{tabular}{|r|r|r|}
\hline
Order of     &   Name      \\
growth       &               \\
\hline
$O(1)$             & constant \\
$O(\log_b n)$      & logarithmic (for any $b$) \\
$O(n)$             & linear \\
$O(n \log_b n)$    & linearithmic \\
$O(n^2)$           & quadratic     \\
$O(n^3)$           & cubic     \\
$O(c^n)$           & exponential (for any $c$)    \\
\hline
\end{tabular}

For the logarithmic terms, the base of the logarithm doesn't matter;
changing bases is the equivalent of multiplying by a constant, which
doesn't change the order of growth.  Similarly, all exponential
functions belong to the same order of growth regardless of the base of
the exponent.
Exponential functions grow very quickly, so exponential algorithms are
only useful for small problems.
\index{logarithmic growth}
\index{exponential growth}


\begin{exercise}

Read the Wikipedia page on Big-Oh notation at
\url{http://en.wikipedia.org/wiki/Big_O_notation} and
answer the following questions:

\begin{enumerate}
\item What is the order of growth of $n^3 + n^2$?
What about $1000000 n^3 + n^2$?
What about $n^3 + 1000000 n^2$?

\item What is the order of growth of $(n^2 + n) \cdot (n + 1)$?  Before
  you start multiplying, remember that you only need the leading term.

\item If $f$ is in $O(g)$, for some unspecified function $g$, what can
  we say about $af+b$?

\item If $f_1$ and $f_2$ are in $O(g)$, what can we say about $f_1 + f_2$?

\item If  $f_1$ is in $O(g)$
and $f_2$ is in $O(h)$,
what can we say about  $f_1 + f_2$?

\item If  $f_1$ is in $O(g)$ and $f_2$ is $O(h)$,
what can we say about  $f_1 \cdot f_2$?
\end{enumerate}

\end{exercise}

Programmers who care about performance often find this kind of
analysis hard to swallow.  They have a point: sometimes the
coefficients and the non-leading terms make a real difference.
Sometimes the details of the hardware, the programming language, and
the characteristics of the input make a big difference.  And for small
problems asymptotic behavior is irrelevant.

But if you keep those caveats in mind, algorithmic analysis is a
useful tool.  At least for large problems, the ``better'' algorithm
is usually better, and sometimes it is {\em much} better.  The
difference between two algorithms with the same order of growth is
usually a constant factor, but the difference between a good algorithm
and a bad algorithm is unbounded!


\section{Analysis of basic Python operations}

In Python, most arithmetic operations are constant time;
multiplication usually takes longer than addition and subtraction, and
division takes even longer, but these run times don't depend on the
magnitude of the operands.  Very large integers are an exception; in
that case the run time increases with the number of digits.
\index{analysis of primitives}

Indexing operations---reading or writing elements in a sequence
or dictionary---are also constant time, regardless of the size
of the data structure.
\index{indexing}

A {\tt for} loop that traverses a sequence or dictionary is
usually linear, as long as all of the operations in the body
of the loop are constant time.  For example, adding up the
elements of a list is linear:

\begin{verbatim}
    total = 0
    for x in t:
        total += x
\end{verbatim}

The built-in function {\tt sum} is also linear because it does
the same thing, but it tends to be faster because it is a more
efficient implementation; in the language of algorithmic analysis,
it has a smaller leading coefficient.

As a rule of thumb, if the body of a loop is in $O(n^a)$ then
the whole loop is in $O(n^{a+1})$.  The exception is if you can
show that the loop exits after a constant number of iterations.
If a loop runs $k$ times regardless of $n$, then
the loop is in $O(n^a)$, even for large $k$.

Multiplying by $k$ doesn't change the order of growth, but neither
does dividing.  So if the body of a loop is in $O(n^a)$ and it runs
$n/k$ times, the loop is in $O(n^{a+1})$, even for large $k$.

Most string and tuple operations are linear, except indexing and {\tt
  len}, which are constant time.  The built-in functions {\tt min} and
{\tt max} are linear.  The run-time of a slice operation is
proportional to the length of the output, but independent of the size
of the input.
\index{string methods}
\index{tuple methods}

String concatenation is linear; the run time depends on the sum
of the lengths of the operands.
\index{string concatenation}

All string methods are linear, but if the lengths of
the strings are bounded by a constant---for example, operations on single
characters---they are considered constant time.
The string method {\tt join} is linear; the run time depends on
the total length of the strings.
\index{join@{\tt join}}

Most list methods are linear, but there are some exceptions:
\index{list methods}

\begin{itemize}

\item Adding an element to the end of a list is constant time on
average; when it runs out of room it occasionally gets copied
to a bigger location, but the total time for $n$ operations
is $O(n)$, so the average time for each
operation is $O(1)$.

\item Removing an element from the end of a list is constant time.

\item Sorting is $O(n \log n)$.
\index{sorting}

\end{itemize}

Most dictionary operations and methods are constant time, but
there are some exceptions:
\index{dictionary methods}

\begin{itemize}

\item The run time of {\tt update} is
  proportional to the size of the dictionary passed as a parameter,
  not the dictionary being updated.

\item {\tt keys}, {\tt values} and {\tt items} are constant time because 
  they return iterators.  But
  if you loop through the iterators, the loop will be linear.
\index{iterator}

\end{itemize}

The performance of dictionaries is one of the minor miracles of
computer science.  We will see how they work in
Section~\ref{hashtable}.


\begin{exercise}

Read the Wikipedia page on sorting algorithms at
\url{http://en.wikipedia.org/wiki/Sorting_algorithm} and answer
the following questions:
\index{sorting}

\begin{enumerate}

\item What is a ``comparison sort?'' What is the best worst-case order
  of growth for a comparison sort?  What is the best worst-case order
  of growth for any sort algorithm?
\index{comparison sort}

\item What is the order of growth of bubble sort, and why does Barack
  Obama think it is ``the wrong way to go?''

\item What is the order of growth of radix sort?  What preconditions
  do we need to use it?

\item What is a stable sort and why might it matter in practice?
\index{stable sort}

\item What is the worst sorting algorithm (that has a name)?

\item What sort algorithm does the C library use?  What sort algorithm
  does Python use?  Are these algorithms stable?  You might have to
  Google around to find these answers.

\item Many of the non-comparison sorts are linear, so why does does
  Python use an $O(n \log n)$ comparison sort?

\end{enumerate}

\end{exercise}


\section{Analysis of search algorithms}

A {\bf search} is an algorithm that takes a collection and a target
item and determines whether the target is in the collection, often
returning the index of the target.
\index{search}

The simplest search algorithm is a ``linear search'', which traverses
the items of the collection in order, stopping if it finds the target.
In the worst case it has to traverse the entire collection, so the run
time is linear.
\index{linear search}

The {\tt in} operator for sequences uses a linear search; so do string
methods like {\tt find} and {\tt count}.
\index{in@{\tt in} operator}

If the elements of the sequence are in order, you can use a {\bf
  bisection search}, which is $O(\log n)$.  Bisection search is
similar to the algorithm you might use to look a word up in a
dictionary (a paper dictionary, not the data structure).  Instead of
starting at the beginning and checking each item in order, you start
with the item in the middle and check whether the word you are looking
for comes before or after.  If it comes before, then you search the
first half of the sequence.  Otherwise you search the second half.
Either way, you cut the number of remaining items in half.
\index{bisection search}

If the sequence has 1,000,000 items, it will take about 20 steps to
find the word or conclude that it's not there.  So that's about 50,000
times faster than a linear search.

Bisection search can be much faster than linear search, but
it requires the sequence to be in order, which might require
extra work.

There is another data structure, called a {\bf hashtable} that
is even faster---it can do a search in constant time---and it
doesn't require the items to be sorted.  Python dictionaries
are implemented using hashtables, which is why most dictionary
operations, including the {\tt in} operator, are constant time.


\section{Hashtables}
\label{hashtable}

To explain how hashtables work and why their performance is so
good, I start with a simple implementation of a map and
gradually improve it until it's a hashtable.
\index{hashtable}

I use Python to demonstrate these implementations, but in real
life you wouldn't write code like this in Python; you would just use a
dictionary!  So for the rest of this chapter, you have to imagine that
dictionaries don't exist and you want to implement a data structure
that maps from keys to values.  The operations you have to
implement are:

\begin{description}

\item[{\tt add(k, v)}:] Add a new item that maps from key {\tt k}
to value {\tt v}.  With a Python dictionary, {\tt d}, this operation
is written {\tt d[k] = v}.

\item[{\tt get(k)}:] Look up and return the value that corresponds
to key {\tt k}.  With a Python dictionary, {\tt d}, this operation
is written {\tt d[k]} or {\tt d.get(k)}.

\end{description}

For now, I assume that each key only appears once.
The simplest implementation of this interface uses a list of
tuples, where each tuple is a key-value pair.
\index{LinearMap@{\tt LinearMap}}

\begin{verbatim}
class LinearMap:

    def __init__(self):
        self.items = []

    def add(self, k, v):
        self.items.append((k, v))

    def get(self, k):
        for key, val in self.items:
            if key == k:
                return val
        raise KeyError
\end{verbatim}

{\tt add} appends a key-value tuple to the list of items, which
takes constant time.

{\tt get} uses a {\tt for} loop to search the list:
if it finds the target key it returns the corresponding value;
otherwise it raises a {\tt KeyError}.
So {\tt get} is linear.
\index{KeyError@{\tt KeyError}}

An alternative is to keep the list sorted by key.  Then {\tt get}
could use a bisection search, which is $O(\log n)$.  But inserting a
new item in the middle of a list is linear, so this might not be the
best option.  There are other data structures that can implement {\tt
  add} and {\tt get} in log time, but that's still not as good as
constant time, so let's move on.
\index{red-black tree}

One way to improve {\tt LinearMap} is to break the list of key-value
pairs into smaller lists.  Here's an implementation called
{\tt BetterMap}, which is a list of 100 LinearMaps.  As we'll see
in a second, the order of growth for {\tt get} is still linear,
but {\tt BetterMap} is a step on the path toward hashtables:
\index{BetterMap@{\tt BetterMap}}

\begin{verbatim}
class BetterMap:

    def __init__(self, n=100):
        self.maps = []
        for i in range(n):
            self.maps.append(LinearMap())

    def find_map(self, k):
        index = hash(k) % len(self.maps)
        return self.maps[index]

    def add(self, k, v):
        m = self.find_map(k)
        m.add(k, v)

    def get(self, k):
        m = self.find_map(k)
        return m.get(k)
\end{verbatim}

\verb"__init__" makes a list of {\tt n} {\tt LinearMap}s.

\verb"find_map" is used by
{\tt add} and {\tt get}
to figure out which map to put the
new item in, or which map to search.

\verb"find_map" uses the built-in function {\tt hash}, which takes
almost any Python object and returns an integer.  A limitation of this
implementation is that it only works with hashable keys.  Mutable
types like lists and dictionaries are unhashable.
\index{hash function}

Hashable objects that are considered equivalent return the same hash
value, but the converse is not necessarily true: two objects with
different values can return the same hash value.

\verb"find_map" uses the modulus operator to wrap the hash values
into the range from 0 to {\tt len(self.maps)}, so the result is a legal
index into the list.  Of course, this means that many different
hash values will wrap onto the same index.  But if the hash function
spreads things out pretty evenly (which is what hash functions
are designed to do), then we expect $n/100$ items per LinearMap.

Since the run time of {\tt LinearMap.get} is proportional to the
number of items, we expect BetterMap to be about 100 times faster
than LinearMap.  The order of growth is still linear, but the
leading coefficient is smaller.  That's nice, but still not
as good as a hashtable.

Here (finally) is the crucial idea that makes hashtables fast: if you
can keep the maximum length of the LinearMaps bounded, {\tt
  LinearMap.get} is constant time.  All you have to do is keep track
of the number of items and when the number of
items per LinearMap exceeds a threshold, resize the hashtable by
adding more LinearMaps.
\index{bounded}

Here is an implementation of a hashtable:
\index{HashMap}

\begin{verbatim}
class HashMap:

    def __init__(self):
        self.maps = BetterMap(2)
        self.num = 0

    def get(self, k):
        return self.maps.get(k)

    def add(self, k, v):
        if self.num == len(self.maps.maps):
            self.resize()

        self.maps.add(k, v)
        self.num += 1

    def resize(self):
        new_maps = BetterMap(self.num * 2)

        for m in self.maps.maps:
            for k, v in m.items:
                new_maps.add(k, v)

        self.maps = new_maps
\end{verbatim}

Each {\tt HashMap} contains a {\tt BetterMap}; \verb"__init__" starts
with just 2 LinearMaps and initializes {\tt num}, which keeps track of
the number of items.

{\tt get} just dispatches to {\tt BetterMap}.  The real work happens
in {\tt add}, which checks the number of items and the size of the
{\tt BetterMap}: if they are equal, the average number of items per
LinearMap is 1, so it calls {\tt resize}.

{\tt resize} make a new {\tt BetterMap}, twice as big as the previous
one, and then ``rehashes'' the items from the old map to the new.

Rehashing is necessary because changing the number of LinearMaps
changes the denominator of the modulus operator in
\verb"find_map".  That means that some objects that used
to hash into the same LinearMap will get split up (which is
what we wanted, right?).
\index{rehashing}

Rehashing is linear, so
{\tt resize} is linear, which might seem bad, since I promised
that {\tt add} would be constant time.  But remember that
we don't have to resize every time, so {\tt add} is usually
constant time and only occasionally linear.  The total amount
of work to run {\tt add} $n$ times is proportional to $n$,
so the average time of each {\tt add} is constant time!
\index{constant time}

To see how this works, think about starting with an empty
HashTable and adding a sequence of items.  We start with 2 LinearMaps,
so the first 2 adds are fast (no resizing required).  Let's
say that they take one unit of work each.  The next add
requires a resize, so we have to rehash the first two
items (let's call that 2 more units of work) and then
add the third item (one more unit).  Adding the next item
costs 1 unit, so the total so far is
6 units of work for 4 items.

The next {\tt add} costs 5 units, but the next three
are only one unit each, so the total is 14 units for the
first 8 adds.

The next {\tt add} costs 9 units, but then we can add 7 more
before the next resize, so the total is 30 units for the
first 16 adds.

After 32 adds, the total cost is 62 units, and I hope you are starting
to see a pattern.  After $n$ adds, where $n$ is a power of two, the
total cost is $2n-2$ units, so the average work per add is
a little less than 2 units.  When $n$ is a power of two, that's
the best case; for other values of $n$ the average work is a little
higher, but that's not important.  The important thing is that it
is $O(1)$.
\index{average cost}

Figure~\ref{fig.hash} shows how this works graphically.  Each
block represents a unit of work.  The columns show the total
work for each add in order from left to right: the first two
{\tt adds} cost 1 units, the third costs 3 units, etc.

\begin{figure}
\centerline{\includegraphics[width=5.5in]{figs/towers.pdf}}
\caption{The cost of a hashtable add.\label{fig.hash}}
\end{figure}

The extra work of rehashing appears as a sequence of increasingly
tall towers with increasing space between them.  Now if you knock
over the towers, spreading the cost of resizing over all
adds, you can see graphically that the total cost after $n$
adds is $2n - 2$.

An important feature of this algorithm is that when we resize the
HashTable it grows geometrically; that is, we multiply the size by a
constant.  If you increase the size
arithmetically---adding a fixed number each time---the average time
per {\tt add} is linear.
\index{geometric resizing}

You can download my implementation of HashMap from
\url{http://thinkpython2.com/code/Map.py}, but remember that there
is no reason to use it; if you want a map, just use a Python dictionary.

\section{Glossary}

\begin{description}

\item[analysis of algorithms:] A way to compare algorithms in terms of
their run time and/or space requirements.
\index{analysis of algorithms}

\item[machine model:] A simplified representation of a computer used
to describe algorithms.
\index{machine model}

\item[worst case:] The input that makes a given algorithm run slowest (or
require the most space.
\index{worst case}

\item[leading term:] In a polynomial, the term with the highest exponent.
\index{leading term}

\item[crossover point:] The problem size where two algorithms require
the same run time or space. 
\index{crossover point}

\item[order of growth:] A set of functions that all grow in a way
considered equivalent for purposes of analysis of algorithms. 
For example, all functions that grow linearly belong to the same
order of growth.
\index{order of growth}

\item[Big-Oh notation:] Notation for representing an order of growth;
for example, $O(n)$ represents the set of functions that grow
linearly. 
\index{Big-Oh notation}

\item[linear:] An algorithm whose run time is proportional to
problem size, at least for large problem sizes.
\index{linear}

\item[quadratic:] An algorithm whose run time is proportional to
$n^2$, where $n$ is a measure of problem size.
\index{quadratic}

\item[search:] The problem of locating an element of a collection
(like a list or dictionary) or determining that it is not present.
\index{search}

\item[hashtable:] A data structure that represents a collection of
key-value pairs and performs search in constant time.
\index{hashtable}

\end{description}
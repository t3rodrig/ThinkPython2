\chapter{Functions}
\label{funcchap}

In the context of programming, a {\bf function} is a named sequence of
statements that performs a computation.  When you define a function,
you specify the name and the sequence of statements.  Later, you can
``call'' the function by name.  
\index{function}

\section{Function calls}
\label{functionchap}
\index{function call}

We have already seen one example of a {\bf function call}:

\begin{verbatim}
>>> type(42)
<class 'int'>
\end{verbatim}
%
The name of the function is {\tt type}.  The expression in parentheses
is called the {\bf argument} of the function.  The result, for this
function, is the type of the argument.
\index{parentheses!argument in}

It is common to say that a function ``takes'' an argument and ``returns''
a result.  The result is also called the {\bf return value}.
\index{argument}
\index{return value}

Python provides functions that convert values
from one type to another.  The {\tt int} function takes any value and
converts it to an integer, if it can, or complains otherwise:
\index{conversion!type}
\index{type conversion}
\index{int function}
\index{function!int}

\begin{verbatim}
>>> int('32')
32
>>> int('Hello')
ValueError: invalid literal for int(): Hello
\end{verbatim}
%
{\tt int} can convert floating-point values to integers, but it
doesn't round off; it chops off the fraction part:

\begin{verbatim}
>>> int(3.99999)
3
>>> int(-2.3)
-2
\end{verbatim}
%
{\tt float} converts integers and strings to floating-point
numbers:
\index{float function}
\index{function!float}

\begin{verbatim}
>>> float(32)
32.0
>>> float('3.14159')
3.14159
\end{verbatim}
%
Finally, {\tt str} converts its argument to a string:
\index{str function}
\index{function!str}

\begin{verbatim}
>>> str(32)
'32'
>>> str(3.14159)
'3.14159'
\end{verbatim}
%

\section{Math functions}
\index{math function}
\index{function, math}

Python has a math module that provides most of the familiar
mathematical functions.  A {\bf module} is a file that contains a
collection of related functions.
\index{module}
\index{module object}

Before we can use the functions in a module, we have to import it with
an {\bf import statement}:

\begin{verbatim}
>>> import math
\end{verbatim}
%
This statement creates a {\bf module object} named math.  If
you display the module object, you get some information about it:

\begin{verbatim}
>>> math
<module 'math' (built-in)>
\end{verbatim}
%
The module object contains the functions and variables defined in the
module.  To access one of the functions, you have to specify the name
of the module and the name of the function, separated by a dot (also
known as a period).  This format is called {\bf dot notation}.
\index{dot notation}

\begin{verbatim}
>>> ratio = signal_power / noise_power
>>> decibels = 10 * math.log10(ratio)

>>> radians = 0.7
>>> height = math.sin(radians)
\end{verbatim}
%
The first example uses \verb"math.log10" to compute 
a signal-to-noise ratio in decibels (assuming that \verb"signal_power" and
\verb"noise_power" are defined).  The math module also provides {\tt log},
which computes logarithms base {\tt e}.
\index{log function}
\index{function!log}
\index{sine function}
\index{radian}
\index{trigonometric function}
\index{function, trigonometric}

The second example finds the sine of {\tt radians}.  The name of the
variable is a hint that {\tt sin} and the other trigonometric
functions ({\tt cos}, {\tt tan}, etc.)  take arguments in radians. To
convert from degrees to radians, divide by 180 and multiply by
$\pi$:

\begin{verbatim}
>>> degrees = 45
>>> radians = degrees / 180.0 * math.pi
>>> math.sin(radians)
0.707106781187
\end{verbatim}
%
The expression {\tt math.pi} gets the variable {\tt pi} from the math
module.  Its value is a floating-point approximation
of $\pi$, accurate to about 15 digits.
\index{pi}

If you know
trigonometry, you can check the previous result by comparing it to
the square root of two divided by two:
\index{sqrt function}
\index{function!sqrt}

\begin{verbatim}
>>> math.sqrt(2) / 2.0
0.707106781187
\end{verbatim}
%

\section{Composition}
\index{composition}

So far, we have looked at the elements of a program---variables,
expressions, and statements---in isolation, without talking about how to
combine them.

One of the most useful features of programming languages is their
ability to take small building blocks and {\bf compose} them.  For
example, the argument of a function can be any kind of expression,
including arithmetic operators:

\begin{verbatim}
x = math.sin(degrees / 360.0 * 2 * math.pi)
\end{verbatim}
%
And even function calls:

\begin{verbatim}
x = math.exp(math.log(x+1))
\end{verbatim}
%
Almost anywhere you can put a value, you can put an arbitrary
expression, with one exception: the left side of an assignment
statement has to be a variable name.  Any other expression on the left
side is a syntax error (we will see exceptions to this rule
later).

\begin{verbatim}
>>> minutes = hours * 60                 # right
>>> hours * 60 = minutes                 # wrong!
SyntaxError: can't assign to operator
\end{verbatim}
%
\index{SyntaxError}
\index{exception!SyntaxError}


\section{Adding new functions}

So far, we have only been using the functions that come with Python,
but it is also possible to add new functions.
A {\bf function definition} specifies the name of a new function and
the sequence of statements that run when the function is called.
\index{function}
\index{function definition}
\index{definition!function}

Here is an example:

\begin{verbatim}
def print_lyrics():
    print("I'm a lumberjack, and I'm okay.")
    print("I sleep all night and I work all day.")
\end{verbatim}
%
{\tt def} is a keyword that indicates that this is a function
definition.  The name of the function is \verb"print_lyrics".  The
rules for function names are the same as for variable names: letters,
numbers and underscore are legal, but the first character
can't be a number.  You can't use a keyword as the name of a function,
and you should avoid having a variable and a function with the same
name.
\index{def keyword}
\index{keyword!def}
\index{argument}

The empty parentheses after the name indicate that this function
doesn't take any arguments.
\index{parentheses!empty}
\index{header}
\index{body}
\index{indentation}
\index{colon}

The first line of the function definition is called the {\bf header};
the rest is called the {\bf body}.  The header has to end with a colon
and the body has to be indented.  By convention, indentation is
always four spaces.  The body can contain
any number of statements.

The strings in the print statements are enclosed in double
quotes.  Single quotes and double quotes do the same thing;
most people use single quotes except in cases like this where
a single quote (which is also an apostrophe) appears in the string.

All quotation marks (single and double)
must be ``straight quotes'', usually
located next to Enter on the keyboard.  ``Curly quotes'', like
the ones in this sentence, are not legal in Python.

If you type a function definition in interactive mode, the interpreter
prints dots ({\tt ...}) to let you know that the definition
isn't complete:
\index{ellipses}

\begin{verbatim}
>>> def print_lyrics():
...     print("I'm a lumberjack, and I'm okay.")
...     print("I sleep all night and I work all day.")
...
\end{verbatim}
%
To end the function, you have to enter an empty line.

Defining a function creates a {\bf function object}, which
has type \verb"function":
\index{function type}
\index{type!function}

\begin{verbatim}
>>> print(print_lyrics)
<function print_lyrics at 0xb7e99e9c>
>>> type(print_lyrics)
<class 'function'>
\end{verbatim}
%
The syntax for calling the new function is the same as
for built-in functions:

\begin{verbatim}
>>> print_lyrics()
I'm a lumberjack, and I'm okay.
I sleep all night and I work all day.
\end{verbatim}
%
Once you have defined a function, you can use it inside another
function.  For example, to repeat the previous refrain, we could write
a function called \verb"repeat_lyrics":

\begin{verbatim}
def repeat_lyrics():
    print_lyrics()
    print_lyrics()
\end{verbatim}
%
And then call \verb"repeat_lyrics":

\begin{verbatim}
>>> repeat_lyrics()
I'm a lumberjack, and I'm okay.
I sleep all night and I work all day.
I'm a lumberjack, and I'm okay.
I sleep all night and I work all day.
\end{verbatim}
%
But that's not really how the song goes.


\section{Definitions and uses}
\index{function definition}

Pulling together the code fragments from the previous section, the
whole program looks like this:

\begin{verbatim}
def print_lyrics():
    print("I'm a lumberjack, and I'm okay.")
    print("I sleep all night and I work all day.")

def repeat_lyrics():
    print_lyrics()
    print_lyrics()

repeat_lyrics()
\end{verbatim}
%
This program contains two function definitions: \verb"print_lyrics" and
\verb"repeat_lyrics".  Function definitions get executed just like other
statements, but the effect is to create function objects.  The statements
inside the function do not run until the function is called, and
the function definition generates no output.
\index{use before def}

As you might expect, you have to create a function before you can
run it.  In other words, the function definition has to run
before the function gets called.

As an exercise, move the last line of this program
to the top, so the function call appears before the definitions. Run 
the program and see what error
message you get.

Now move the function call back to the bottom
and move the definition of \verb"print_lyrics" after the definition of
\verb"repeat_lyrics".  What happens when you run this program?


\section{Flow of execution}
\index{flow of execution}

To ensure that a function is defined before its first use,
you have to know the order statements run in, which is
called the {\bf flow of execution}.

Execution always begins at the first statement of the program.
Statements are run one at a time, in order from top to bottom.

Function definitions do not alter the flow of execution of the
program, but remember that statements inside the function don't
run until the function is called.

A function call is like a detour in the flow of execution. Instead of
going to the next statement, the flow jumps to the body of
the function, runs the statements there, and then comes back
to pick up where it left off.

That sounds simple enough, until you remember that one function can
call another.  While in the middle of one function, the program might
have to run the statements in another function.  Then, while
running that new function, the program might have to run yet
another function!

Fortunately, Python is good at keeping track of where it is, so each
time a function completes, the program picks up where it left off in
the function that called it.  When it gets to the end of the program,
it terminates.

In summary, when you read a program, you
don't always want to read from top to bottom.  Sometimes it makes
more sense if you follow the flow of execution.


\section{Parameters and arguments}
\label{parameters}
\index{parameter}
\index{function parameter}
\index{argument}
\index{function argument}

Some of the functions we have seen require arguments.  For
example, when you call {\tt math.sin} you pass a number
as an argument.  Some functions take more than one argument:
{\tt math.pow} takes two, the base and the exponent.

Inside the function, the arguments are assigned to
variables called {\bf parameters}.  Here is a definition for
a function that takes an argument:
\index{parentheses!parameters in}

\begin{verbatim}
def print_twice(bruce):
    print(bruce)
    print(bruce)
\end{verbatim}
%
This function assigns the argument to a parameter
named {\tt bruce}.  When the function is called, it prints the value of
the parameter (whatever it is) twice.

This function works with any value that can be printed.

\begin{verbatim}
>>> print_twice('Spam')
Spam
Spam
>>> print_twice(42)
42
42
>>> print_twice(math.pi)
3.14159265359
3.14159265359
\end{verbatim}
%
The same rules of composition that apply to built-in functions also
apply to programmer-defined functions, so we can use any kind of expression
as an argument for \verb"print_twice":
\index{composition}
\index{programmer-defined function}
\index{function!programmer defined}

\begin{verbatim}
>>> print_twice('Spam '*4)
Spam Spam Spam Spam
Spam Spam Spam Spam
>>> print_twice(math.cos(math.pi))
-1.0
-1.0
\end{verbatim}
%
The argument is evaluated before the function is called, so
in the examples the expressions \verb"'Spam '*4" and
{\tt math.cos(math.pi)} are only evaluated once.
\index{argument}

You can also use a variable as an argument:

\begin{verbatim}
>>> michael = 'Eric, the half a bee.'
>>> print_twice(michael)
Eric, the half a bee.
Eric, the half a bee.
\end{verbatim}
%
The name of the variable we pass as an argument ({\tt michael}) has
nothing to do with the name of the parameter ({\tt bruce}).  It
doesn't matter what the value was called back home (in the caller);
here in \verb"print_twice", we call everybody {\tt bruce}.


\section{Variables and parameters are local}
\index{local variable}
\index{variable!local}

When you create a variable inside a function, it is {\bf local},
which means that it only
exists inside the function.  For example:
\index{parentheses!parameters in}

\begin{verbatim}
def cat_twice(part1, part2):
    cat = part1 + part2
    print_twice(cat)
\end{verbatim}
%
This function takes two arguments, concatenates them, and prints
the result twice.  Here is an example that uses it:
\index{concatenation}

\begin{verbatim}
>>> line1 = 'Bing tiddle '
>>> line2 = 'tiddle bang.'
>>> cat_twice(line1, line2)
Bing tiddle tiddle bang.
Bing tiddle tiddle bang.
\end{verbatim}
%
When \verb"cat_twice" terminates, the variable {\tt cat}
is destroyed.  If we try to print it, we get an exception:
\index{NameError}
\index{exception!NameError}

\begin{verbatim}
>>> print(cat)
NameError: name 'cat' is not defined
\end{verbatim}
%
Parameters are also local.
For example, outside \verb"print_twice", there is no
such thing as {\tt bruce}.
\index{parameter}


\section{Stack diagrams}
\label{stackdiagram}
\index{stack diagram}
\index{function frame}
\index{frame}

To keep track of which variables can be used where, it is sometimes
useful to draw a {\bf stack diagram}.  Like state diagrams, stack
diagrams show the value of each variable, but they also show the
function each variable belongs to.
\index{stack diagram}
\index{diagram!stack}

Each function is represented by a {\bf frame}.  A frame is a box with
the name of a function beside it and the parameters and variables of
the function inside it.  The stack diagram for the previous example is
shown in Figure~\ref{fig.stack}.

\begin{figure}
\centerline
{\includegraphics[scale=0.8]{figs/stack.pdf}}
\caption{Stack diagram.}
\label{fig.stack}
\end{figure}


The frames are arranged in a stack that indicates which function
called which, and so on.  In this example, \verb"print_twice"
was called by \verb"cat_twice", and \verb"cat_twice" was called by 
\verb"__main__", which is a special name for the topmost frame.  When
you create a variable outside of any function, it belongs to 
\verb"__main__".

\index{main}

Each parameter refers to the same value as its corresponding
argument.  So, {\tt part1} has the same value as
{\tt line1}, {\tt part2} has the same value as {\tt line2},
and {\tt bruce} has the same value as {\tt cat}.

If an error occurs during a function call, Python prints the
name of the function, the name of the function that called
it, and the name of the function that called {\em that}, all the
way back to \verb"__main__".

For example, if you try to access {\tt cat} from within 
\verb"print_twice", you get a {\tt NameError}:

\begin{verbatim}
Traceback (innermost last):
  File "test.py", line 13, in __main__
    cat_twice(line1, line2)
  File "test.py", line 5, in cat_twice
    print_twice(cat)
  File "test.py", line 9, in print_twice
    print(cat)
NameError: name 'cat' is not defined
\end{verbatim}
%
This list of functions is called a {\bf traceback}.  It tells you what
program file the error occurred in, and what line, and what functions
were executing at the time.  It also shows the line of code that
caused the error.
\index{traceback}

The order of the functions in the traceback is the same as the
order of the frames in the stack diagram.  The function that is
currently running is at the bottom.


\section{Fruitful functions and void functions}
\index{fruitful function}
\index{void function}
\index{function, fruitful}
\index{function, void} 

Some of the functions we have used, such as the math functions, return
results; for lack of a better name, I call them {\bf fruitful
  functions}.  Other functions, like \verb"print_twice", perform an
action but don't return a value.  They are called {\bf void
  functions}.

When you call a fruitful function, you almost always
want to do something with the result; for example, you might
assign it to a variable or use it as part of an expression:

\begin{verbatim}
x = math.cos(radians)
golden = (math.sqrt(5) + 1) / 2
\end{verbatim}
%
When you call a function in interactive mode, Python displays
the result:

\begin{verbatim}
>>> math.sqrt(5)
2.2360679774997898
\end{verbatim}
%
But in a script, if you call a fruitful function all by itself,
the return value is lost forever!

\begin{verbatim}
math.sqrt(5)
\end{verbatim}
%
This script computes the square root of 5, but since it doesn't store
or display the result, it is not very useful.
\index{interactive mode}
\index{script mode}

Void functions might display something on the screen or have some
other effect, but they don't have a return value.  If you
assign the result to a variable, you get a special value called
{\tt None}.
\index{None special value}
\index{special value!None}

\begin{verbatim}
>>> result = print_twice('Bing')
Bing
Bing
>>> print(result)
None
\end{verbatim}
%
The value {\tt None} is not the same as the string \verb"'None'". 
It is a special value that has its own type:

\begin{verbatim}
>>> type(None)
<class 'NoneType'>
\end{verbatim}
%
The functions we have written so far are all void.  We will start
writing fruitful functions in a few chapters.
\index{NoneType type}
\index{type!NoneType}


\section{Why functions?}
\index{function, reasons for}

It may not be clear why it is worth the trouble to divide
a program into functions.  There are several reasons:

\begin{itemize}

\item Creating a new function gives you an opportunity to name a group
of statements, which makes your program easier to read and debug.

\item Functions can make a program smaller by eliminating repetitive
code.  Later, if you make a change, you only have
to make it in one place.

\item Dividing a long program into functions allows you to debug the
parts one at a time and then assemble them into a working whole.

\item Well-designed functions are often useful for many programs.
Once you write and debug one, you can reuse it.

\end{itemize}


\section{Debugging}

One of the most important skills you will acquire is debugging.
Although it can be frustrating, debugging is one of the most
intellectually rich, challenging, and interesting parts of
programming.
\index{experimental debugging}
\index{debugging!experimental}

In some ways debugging is like detective work.  You are confronted
with clues and you have to infer the processes and events that led
to the results you see.

Debugging is also like an experimental science.  Once you have an idea
about what is going wrong, you modify your program and try again.  If
your hypothesis was correct, you can predict the result of the
modification, and you take a step closer to a working program.  If
your hypothesis was wrong, you have to come up with a new one.  As
Sherlock Holmes pointed out, ``When you have eliminated the
impossible, whatever remains, however improbable, must be the truth.''
(A. Conan Doyle, {\em The Sign of Four})
\index{Holmes, Sherlock}
\index{Doyle, Arthur Conan}

For some people, programming and debugging are the same thing.  That
is, programming is the process of gradually debugging a program until
it does what you want.  The idea is that you should start with a
working program and make small modifications,
debugging them as you go.

For example, Linux is an operating system that contains millions of
lines of code, but it started out as a simple program Linus Torvalds
used to explore the Intel 80386 chip.  According to Larry Greenfield,
``One of Linus's earlier projects was a program that would switch
between printing AAAA and BBBB.  This later evolved to Linux.''
({\em The Linux Users' Guide} Beta Version 1).
\index{Linux}


\section{Glossary}

\begin{description}

\item[function:] A named sequence of statements that performs some
useful operation.  Functions may or may not take arguments and may or
may not produce a result.
\index{function}

\item[function definition:]  A statement that creates a new function,
specifying its name, parameters, and the statements it contains.
\index{function definition}

\item[function object:]  A value created by a function definition.
The name of the function is a variable that refers to a function
object.
\index{function definition}

\item[header:] The first line of a function definition.
\index{header}

\item[body:] The sequence of statements inside a function definition.
\index{body}

\item[parameter:] A name used inside a function to refer to the value
passed as an argument.
\index{parameter}

\item[function call:] A statement that runs a function. It
consists of the function name followed by an argument list in
parentheses.
\index{function call}

\item[argument:]  A value provided to a function when the function is called.
This value is assigned to the corresponding parameter in the function.
\index{argument}

\item[local variable:]  A variable defined inside a function.  A local
variable can only be used inside its function.
\index{local variable}

\item[return value:]  The result of a function.  If a function call
is used as an expression, the return value is the value of
the expression.
\index{return value}

\item[fruitful function:] A function that returns a value.
\index{fruitful function}

\item[void function:] A function that always returns {\tt None}.
\index{void function}

\item[{\tt None}:]  A special value returned by void functions.
\index{None special value}
\index{special value!None}

\item[module:] A file that contains a
collection of related functions and other definitions.
\index{module}

\item[import statement:] A statement that reads a module file and creates
a module object.
\index{import statement}
\index{statement!import}

\item[module object:] A value created by an {\tt import} statement
that provides access to the values defined in a module.
\index{module}

\item[dot notation:]  The syntax for calling a function in another
module by specifying the module name followed by a dot (period) and
the function name.
\index{dot notation}

\item[composition:] Using an expression as part of a larger expression,
or a statement as part of a larger statement.
\index{composition}

\item[flow of execution:]  The order statements run in.
\index{flow of execution}

\item[stack diagram:]  A graphical representation of a stack of functions,
their variables, and the values they refer to.
\index{stack diagram}

\item[frame:]  A box in a stack diagram that represents a function call.
It contains the local variables and parameters of the function.
\index{function frame}
\index{frame}

\item[traceback:]  A list of the functions that are executing,
printed when an exception occurs.
\index{traceback}


\end{description}


\section{Exercises}

\begin{exercise}
\index{len function}
\index{function!len}

Write a function named \verb"right_justify" that takes a string
named {\tt s} as a parameter and prints the string with enough
leading spaces so that the last letter of the string is in column 70
of the display.

\begin{verbatim}
>>> right_justify('monty')
                                                                 monty
\end{verbatim}

Hint: Use string concatenation and repetition.  Also,
Python provides a built-in function called {\tt len} that
returns the length of a string, so the value of \verb"len('monty')" is 5.

\end{exercise}


\begin{exercise}
\index{function object}
\index{object!function}

A function object is a value you can assign to a variable
or pass as an argument.  For example, \verb"do_twice" is a function
that takes a function object as an argument and calls it twice:

\begin{verbatim}
def do_twice(f):
    f()
    f()
\end{verbatim}

Here's an example that uses \verb"do_twice" to call a function
named \verb"print_spam" twice.

\begin{verbatim}
def print_spam():
    print('spam')

do_twice(print_spam)
\end{verbatim}

\begin{enumerate}

\item Type this example into a script and test it.

\item Modify \verb"do_twice" so that it takes two arguments, a
function object and a value, and calls the function twice,
passing the value as an argument.

\item Copy the definition of 
\verb"print_twice" from earlier in this chapter to your script.

\item Use the modified version of \verb"do_twice" to call
\verb"print_twice" twice, passing \verb"'spam'" as an argument.

\item Define a new function called 
\verb"do_four" that takes a function object and a value
and calls the function four times, passing the value
as a parameter.  There should be only
two statements in the body of this function, not four.

\end{enumerate}

Solution: \url{http://thinkpython2.com/code/do_four.py}.

\end{exercise}



\begin{exercise}

Note: This exercise should be
done using only the statements and other features we have learned so
far.  

\begin{enumerate}

\item Write a function that draws a grid like the following:
\index{grid}

\begin{verbatim}
+ - - - - + - - - - +
|         |         |
|         |         |
|         |         |
|         |         |
+ - - - - + - - - - +
|         |         |
|         |         |
|         |         |
|         |         |
+ - - - - + - - - - +
\end{verbatim}
%
Hint: to print more than one value on a line, you can print
a comma-separated sequence of values:

\begin{verbatim}
print('+', '-')
\end{verbatim}
%
By default, {\tt print} advances to the next line, but you
can override that behavior and put a space at the end, like this:

\begin{verbatim}
print('+', end=' ')
print('-')
\end{verbatim}
%
The output of these statements is \verb"'+ -'" on the same line.
The output from the next print statement would begin on the next line.

\item Write a function that draws a similar grid
with four rows and four columns.

\end{enumerate}

Solution: \url{http://thinkpython2.com/code/grid.py}.
Credit: This exercise is based on an exercise in Oualline, {\em
    Practical C Programming, Third Edition}, O'Reilly Media, 1997.

\end{exercise}
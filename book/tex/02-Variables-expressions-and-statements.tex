\chapter{Variables, expressions and statements}

One of the most powerful features of a programming language is the
ability to manipulate {\bf variables}.  A variable is a name that
refers to a value.
\index{variable}


\section{Assignment statements}
\label{variables}
\index{assignment statement}
\index{statement!assignment}

An {\bf assignment statement} creates a new variable and gives
it a value:

\begin{verbatim}
>>> message = 'And now for something completely different'
>>> n = 17
>>> pi = 3.141592653589793
\end{verbatim}
%
This example makes three assignments.  The first assigns a string
to a new variable named {\tt message};
the second gives the integer {\tt 17} to {\tt n}; the third
assigns the (approximate) value of $\pi$ to {\tt pi}.
\index{state diagram}
\index{diagram!state}

A common way to represent variables on paper is to write the name with
an arrow pointing to its value.  This kind of figure is
called a {\bf state diagram} because it shows what state each of the
variables is in (think of it as the variable's state of mind).
Figure~\ref{fig.state2} shows the result of the previous example.

\begin{figure}
\centerline
{\includegraphics[scale=0.8]{figs/state2.pdf}}
\caption{State diagram.}
\label{fig.state2}
\end{figure}



\section{Variable names}
\index{variable}

Programmers generally choose names for their variables that
are meaningful---they document what the variable is used for.

Variable names can be as long as you like.  They can contain
both letters and numbers, but they can't begin with a number.
It is legal to use uppercase letters, but it is conventional
to use only lower case for variables names.

The underscore character, \verb"_", can appear in a name.
It is often used in names with multiple words, such as
\verb"your_name" or \verb"airspeed_of_unladen_swallow".
\index{underscore character}

If you give a variable an illegal name, you get a syntax error:

\begin{verbatim}
>>> 76trombones = 'big parade'
SyntaxError: invalid syntax
>>> more@ = 1000000
SyntaxError: invalid syntax
>>> class = 'Advanced Theoretical Zymurgy'
SyntaxError: invalid syntax
\end{verbatim}
%
{\tt 76trombones} is illegal because it begins with a number.
{\tt more@} is illegal because it contains an illegal character, {\tt
@}.  But what's wrong with {\tt class}?

It turns out that {\tt class} is one of Python's {\bf keywords}.  The
interpreter uses keywords to recognize the structure of the program,
and they cannot be used as variable names.
\index{keyword}

Python 3 has these keywords:

\begin{verbatim}
False      class      finally    is         return
None       continue   for        lambda     try
True       def        from       nonlocal   while
and        del        global     not        with
as         elif       if         or         yield
assert     else       import     pass
break      except     in         raise
\end{verbatim}
%
You don't have to memorize this list.  In most development environments,
keywords are displayed in a different color; if you try to use one
as a variable name, you'll know.


\section{Expressions and statements}

An {\bf expression} is a combination of values, variables, and operators.
A value all by itself is considered an expression, and so is
a variable, so the following are all legal expressions:
\index{expression}

\begin{verbatim}
>>> 42
42
>>> n
17
>>> n + 25
42
\end{verbatim}
%
When you type an expression at the prompt, the interpreter
{\bf evaluates} it, which means that it finds the value of
the expression.
In this example, {\tt n} has the value 17 and
{\tt n + 25} has the value 42.
\index{evaluate}

A {\bf statement} is a unit of code that has an effect, like
creating a variable or displaying a value.  
\index{statement}

\begin{verbatim}
>>> n = 17
>>> print(n)
\end{verbatim}
%
The first line is an assignment statement that gives a value to
{\tt n}.  The second line is a print statement that displays the
value of {\tt n}.

When you type a statement, the interpreter {\bf executes} it,
which means that it does whatever the statement says.  In general,
statements don't have values.
\index{execute}


\section{Script mode}

So far we have run Python in {\bf interactive mode}, which
means that you interact directly with the interpreter.
Interactive mode is a good way to get started,
but if you are working with more than a few lines of code, it can be
clumsy.
\index{interactive mode}

The alternative is to save code in a file called a {\bf script} and
then run the interpreter in {\bf script mode} to execute the script.  By
convention, Python scripts have names that end with {\tt .py}.
\index{script}
\index{script mode}

If you know how to create and run a script on your computer, you
are ready to go.  Otherwise I recommend using PythonAnywhere again.
I have posted instructions for running in script mode at
\url{http://tinyurl.com/thinkpython2e}.

Because Python provides both modes,
you can test bits of code in interactive mode before you put them
in a script.  But there are differences between interactive mode
and script mode that can be confusing.
\index{interactive mode}
\index{script mode}

For example, if you are using Python as a calculator, you might type

\begin{verbatim}
>>> miles = 26.2
>>> miles * 1.61
42.182
\end{verbatim}

The first line assigns a value to {\tt miles}, but it has no visible
effect.  The second line is an expression, so the
interpreter evaluates it and displays the result.  It turns out that a
marathon is about 42 kilometers.

But if you type the same code into a script and run it, you get no
output at all.  In script mode an expression, all by itself, has no
visible effect.  Python actually evaluates the expression, but it doesn't
display the value unless you tell it to:

\begin{verbatim}
miles = 26.2
print(miles * 1.61)
\end{verbatim}

This behavior can be confusing at first.

A script usually contains a sequence of statements.  If there
is more than one statement, the results appear one at a time
as the statements execute.

For example, the script

\begin{verbatim}
print(1)
x = 2
print(x)
\end{verbatim}
%
produces the output

\begin{verbatim}
1
2
\end{verbatim}
%
The assignment statement produces no output.

To check your understanding, type the following statements in the
Python interpreter and see what they do:

\begin{verbatim}
5
x = 5
x + 1
\end{verbatim}

Now put the same statements in a script and run it.  What
is the output?  Modify the script by transforming each
expression into a print statement and then run it again.



\section{Order of operations}
\index{order of operations}
\index{PEMDAS}

When an expression contains more than one operator, the order of
evaluation depends on the {\bf order of operations}.  For
mathematical operators, Python follows mathematical convention.
The acronym {\bf PEMDAS} is a useful way to
remember the rules:

\begin{itemize}

\item {\bf P}arentheses have the highest precedence and can be used 
to force an expression to evaluate in the order you want. Since
expressions in parentheses are evaluated first, {\tt 2 * (3-1)} is 4,
and {\tt (1+1)**(5-2)} is 8. You can also use parentheses to make an
expression easier to read, as in {\tt (minute * 100) / 60}, even
if it doesn't change the result.

\item {\bf E}xponentiation has the next highest precedence, so
{\tt 1 + 2**3} is 9, not 27, and {\tt 2 * 3**2} is 18, not 36.

\item {\bf M}ultiplication and {\bf D}ivision have higher precedence
  than {\bf A}ddition and {\bf S}ubtraction.  So {\tt 2*3-1} is 5, not
  4, and {\tt 6+4/2} is 8, not 5.

\item Operators with the same precedence are evaluated from left to
  right (except exponentiation).  So in the expression {\tt degrees /
    2 * pi}, the division happens first and the result is multiplied
  by {\tt pi}.  To divide by $2 \pi$, you can use parentheses or write
  {\tt degrees / 2 / pi}.

\end{itemize}

I don't work very hard to remember the precedence of
operators.  If I can't tell by looking at the expression, I use
parentheses to make it obvious.


\section{String operations}
\index{string!operation}
\index{operator!string}

In general, you can't perform mathematical operations on strings, even
if the strings look like numbers, so the following are illegal:

\begin{verbatim}
'2'-'1'    'eggs'/'easy'    'third'*'a charm'
\end{verbatim}
%
But there are two exceptions, {\tt +} and {\tt *}.

The {\tt +} operator performs {\bf string concatenation}, which means
it joins the strings by linking them end-to-end.  For example:
\index{concatenation}

\begin{verbatim}
>>> first = 'throat'
>>> second = 'warbler'
>>> first + second
throatwarbler
\end{verbatim}
%
The {\tt *} operator also works on strings; it performs repetition.
For example, \verb"'Spam'*3" is \verb"'SpamSpamSpam'".  If one of the
values is a string, the other has to be an integer.

This use of {\tt +} and {\tt *} makes sense by
analogy with addition and multiplication.  Just as {\tt 4*3} is
equivalent to {\tt 4+4+4}, we expect \verb"'Spam'*3" to be the same as
\verb"'Spam'+'Spam'+'Spam'", and it is.  On the other hand, there is a
significant way in which string concatenation and repetition are
different from integer addition and multiplication.
Can you think of a property that addition has
that string concatenation does not?
\index{commutativity}


\section{Comments}
\index{comment}

As programs get bigger and more complicated, they get more difficult
to read.  Formal languages are dense, and it is often difficult to
look at a piece of code and figure out what it is doing, or why.

For this reason, it is a good idea to add notes to your programs to explain
in natural language what the program is doing.  These notes are called
{\bf comments}, and they start with the \verb"#" symbol:

\begin{verbatim}
# compute the percentage of the hour that has elapsed
percentage = (minute * 100) / 60
\end{verbatim}
%
In this case, the comment appears on a line by itself.  You can also put
comments at the end of a line:

\begin{verbatim}
percentage = (minute * 100) / 60     # percentage of an hour
\end{verbatim}
%
Everything from the {\tt \#} to the end of the line is ignored---it
has no effect on the execution of the program.

Comments are most useful when they document non-obvious features of
the code.  It is reasonable to assume that the reader can figure out
{\em what} the code does; it is more useful to explain {\em why}.

This comment is redundant with the code and useless:

\begin{verbatim}
v = 5     # assign 5 to v
\end{verbatim}
%
This comment contains useful information that is not in the code:

\begin{verbatim}
v = 5     # velocity in meters/second. 
\end{verbatim}
%
Good variable names can reduce the need for comments, but
long names can make complex expressions hard to read, so there is
a tradeoff.


\section{Debugging}
\index{debugging}
\index{bug}

Three kinds of errors can occur in a program: syntax errors, runtime 
errors, and semantic errors.  It is useful
to distinguish between them in order to track them down more quickly.

\begin{description}

\item[Syntax error:] ``Syntax'' refers to the structure of a program
  and the rules about that structure.  For example, parentheses have
  to come in matching pairs, so {\tt (1 + 2)} is legal, but {\tt 8)}
  is a {\bf syntax error}.  \index{syntax error} \index{error!syntax}
  \index{error message}
\index{syntax} 

If there is a syntax error
anywhere in your program, Python displays an error message and quits,
and you will not be able to run the program.  During the first few
weeks of your programming career, you might spend a lot of
time tracking down syntax errors.  As you gain experience, you will
make fewer errors and find them faster.


\item[Runtime error:] The second type of error is a runtime error, so
  called because the error does not appear until after the program has
  started running.  These errors are also called {\bf exceptions}
  because they usually indicate that something exceptional (and bad)
  has happened.  \index{runtime error} \index{error!runtime}
  \index{exception} \index{safe language} \index{language!safe}

Runtime errors are rare in the simple programs you will see in the
first few chapters, so it might be a while before you encounter one.


\item[Semantic error:] The third type of error is ``semantic'', which
  means related to meaning.  If there is a semantic error in your
  program, it will run without generating error messages, but it will
  not do the right thing.  It will do something else.  Specifically,
  it will do what you told it to do.  \index{semantic error}
  \index{error!semantic} \index{error message}

Identifying semantic errors can be tricky because it requires you to work
backward by looking at the output of the program and trying to figure
out what it is doing.

\end{description}


\section{Glossary}

\begin{description}

\item[variable:]  A name that refers to a value.
\index{variable}

\item[assignment:]  A statement that assigns a value to a variable.
\index{assignment}

\item[state diagram:]  A graphical representation of a set of variables and the
values they refer to.
\index{state diagram}

\item[keyword:]  A reserved word that is used to parse a
program; you cannot use keywords like {\tt if}, {\tt  def}, and {\tt while} as
variable names.
\index{keyword}

\item[operand:]  One of the values on which an operator operates.
\index{operand}

\item[expression:]  A combination of variables, operators, and values that
represents a single result.
\index{expression}

\item[evaluate:]  To simplify an expression by performing the operations
in order to yield a single value.

\item[statement:]  A section of code that represents a command or action.  So
far, the statements we have seen are assignments and print statements.
\index{statement}

\item[execute:]  To run a statement and do what it says.
\index{execute}

\item[interactive mode:] A way of using the Python interpreter by
typing code at the prompt.
\index{interactive mode}

\item[script mode:] A way of using the Python interpreter to read
code from a script and run it.
\index{script mode}

\item[script:] A program stored in a file.
\index{script}

\item[order of operations:]  Rules governing the order in which
expressions involving multiple operators and operands are evaluated.
\index{order of operations}

\item[concatenate:]  To join two operands end-to-end.
\index{concatenation}

\item[comment:]  Information in a program that is meant for other
programmers (or anyone reading the source code) and has no effect on the
execution of the program.
\index{comment}

\item[syntax error:]  An error in a program that makes it impossible
to parse (and therefore impossible to interpret).
\index{syntax error}

\item[exception:]  An error that is detected while the program is running.
\index{exception}

\item[semantics:]  The meaning of a program.
\index{semantics}

\item[semantic error:]   An error in a program that makes it do something
other than what the programmer intended.
\index{semantic error}

\end{description}


\section{Exercises}

\begin{exercise}

Repeating my advice from the previous chapter, whenever you learn
a new feature, you should try it out in interactive mode and make
errors on purpose to see what goes wrong.

\begin{itemize}

\item We've seen that {\tt n = 42} is legal.  What about {\tt 42 = n}?

\item How about {\tt x = y = 1}?

\item In some languages every statement ends with a semi-colon, {\tt ;}.
What happens if you put a semi-colon at the end of a Python statement?

\item What if you put a period at the end of a statement?

\item In math notation you can multiply $x$ and $y$ like this: $x y$.
What happens if you try that in Python?

\end{itemize}

\end{exercise}


\begin{exercise}

Practice using the Python interpreter as a calculator: 
\index{calculator}

\begin{enumerate}

\item The volume of a sphere with radius $r$ is $\frac{4}{3} \pi r^3$.
  What is the volume of a sphere with radius 5?

\item Suppose the cover price of a book is \$24.95, but bookstores get a
  40\% discount.  Shipping costs \$3 for the first copy and 75 cents
  for each additional copy.  What is the total wholesale cost for
  60 copies?

\item If I leave my house at 6:52 am and run 1 mile at an easy pace
  (8:15 per mile), then 3 miles at tempo (7:12 per mile) and 1 mile at
  easy pace again, what time do I get home for breakfast?
\index{running pace}

\end{enumerate}
\end{exercise}